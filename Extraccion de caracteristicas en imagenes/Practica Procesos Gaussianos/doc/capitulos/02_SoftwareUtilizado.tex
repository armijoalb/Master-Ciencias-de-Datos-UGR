\chapter{Software Utilizado para el desarrollo de la práctica}
El software elegido para desarrollar esta práctica ha sido \textbf{GPflow} y el lenguaje de programación \textbf{Python}. Este paquete permite crear modelos de Procesos Gaussianos utilizando \textbf{tensorflow} para calcular la optimización de los parámetros. Para crear un modelo como el que se pide en la práctica debemos utilizar el modelo \textit{gpflow.models.VGP}; este modelo calcula un proceso gaussiano variacional, que es lo que se ha estudiado en la asignatura. A dicho modelo deberemos de pasarle los datos de train, la etiquetas de los datos, el kernel que utilizará el modelo y la distribución de probabilidad de las etiquetes (likelihood). \newline

El parámetro \textit{likelihood} del modelo será en nuestro caso siempre un objeto de la clase \textit{gpflow.likelihoods.Bernoulli()}, dicha función utiliza internamente una distribución igual que la logística definida en la práctica. Los kernel utilizados son \textit{gpflow.kernels.RBF} y \textit{gpflow.kernels.Linear}, a cada uno de los kernels se le debe indicar el número de dimensiones que tienen los datos; al kernel radial (RBF) se le debe pasar también los parámetros \textit{variance} y \textit{lengthscale} iniciales.\newline

Una vez se ha creado el modelo, se deben optimizar los parámetros para ajustar el modelo a los datos, para ello se debe utilizar la función \textit{gpflow.train.ScipyOptimizer().minimize}; a esta función se le debe pasar el modelo y opcionalmente el número de iteraciones máximo que debe realizar, en la práctica se ha utilizado un máximo de 250 iteraciones ya que a partir de dicho número los resultados obtenidos no mejoraban.\newline

Adicionalmente, se han utilizado las librerías \textbf{sklearn}, \textbf{numpy}, \textbf{scipy.io}, \textbf{seaborn}, \textbf{pandas}, \textbf{matplotlib} para la lectura de datos, dibujar las diferentes gráficas y el cálculo de métricas.