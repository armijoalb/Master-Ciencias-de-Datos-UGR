
% Default to the notebook output style

    


% Inherit from the specified cell style.




    
\documentclass[11pt]{article}

    
    
    \usepackage[T1]{fontenc}
    % Nicer default font (+ math font) than Computer Modern for most use cases
    \usepackage{mathpazo}

    % Basic figure setup, for now with no caption control since it's done
    % automatically by Pandoc (which extracts ![](path) syntax from Markdown).
    \usepackage{graphicx}
    % We will generate all images so they have a width \maxwidth. This means
    % that they will get their normal width if they fit onto the page, but
    % are scaled down if they would overflow the margins.
    \makeatletter
    \def\maxwidth{\ifdim\Gin@nat@width>\linewidth\linewidth
    \else\Gin@nat@width\fi}
    \makeatother
    \let\Oldincludegraphics\includegraphics
    % Set max figure width to be 80% of text width, for now hardcoded.
    \renewcommand{\includegraphics}[1]{\Oldincludegraphics[width=.8\maxwidth]{#1}}
    % Ensure that by default, figures have no caption (until we provide a
    % proper Figure object with a Caption API and a way to capture that
    % in the conversion process - todo).
    \usepackage{caption}
    \DeclareCaptionLabelFormat{nolabel}{}
    \captionsetup{labelformat=nolabel}

    \usepackage{adjustbox} % Used to constrain images to a maximum size 
    \usepackage{xcolor} % Allow colors to be defined
    \usepackage{enumerate} % Needed for markdown enumerations to work
    \usepackage{geometry} % Used to adjust the document margins
    \usepackage{amsmath} % Equations
    \usepackage{amssymb} % Equations
    \usepackage{textcomp} % defines textquotesingle
    % Hack from http://tex.stackexchange.com/a/47451/13684:
    \AtBeginDocument{%
        \def\PYZsq{\textquotesingle}% Upright quotes in Pygmentized code
    }
    \usepackage{upquote} % Upright quotes for verbatim code
    \usepackage{eurosym} % defines \euro
    \usepackage[mathletters]{ucs} % Extended unicode (utf-8) support
    \usepackage[utf8x]{inputenc} % Allow utf-8 characters in the tex document
    \usepackage{fancyvrb} % verbatim replacement that allows latex
    \usepackage{grffile} % extends the file name processing of package graphics 
                         % to support a larger range 
    % The hyperref package gives us a pdf with properly built
    % internal navigation ('pdf bookmarks' for the table of contents,
    % internal cross-reference links, web links for URLs, etc.)
    \usepackage{hyperref}
    \usepackage{longtable} % longtable support required by pandoc >1.10
    \usepackage{booktabs}  % table support for pandoc > 1.12.2
    \usepackage[inline]{enumitem} % IRkernel/repr support (it uses the enumerate* environment)
    \usepackage[normalem]{ulem} % ulem is needed to support strikethroughs (\sout)
                                % normalem makes italics be italics, not underlines
    

    
    
    % Colors for the hyperref package
    \definecolor{urlcolor}{rgb}{0,.145,.698}
    \definecolor{linkcolor}{rgb}{.71,0.21,0.01}
    \definecolor{citecolor}{rgb}{.12,.54,.11}

    % ANSI colors
    \definecolor{ansi-black}{HTML}{3E424D}
    \definecolor{ansi-black-intense}{HTML}{282C36}
    \definecolor{ansi-red}{HTML}{E75C58}
    \definecolor{ansi-red-intense}{HTML}{B22B31}
    \definecolor{ansi-green}{HTML}{00A250}
    \definecolor{ansi-green-intense}{HTML}{007427}
    \definecolor{ansi-yellow}{HTML}{DDB62B}
    \definecolor{ansi-yellow-intense}{HTML}{B27D12}
    \definecolor{ansi-blue}{HTML}{208FFB}
    \definecolor{ansi-blue-intense}{HTML}{0065CA}
    \definecolor{ansi-magenta}{HTML}{D160C4}
    \definecolor{ansi-magenta-intense}{HTML}{A03196}
    \definecolor{ansi-cyan}{HTML}{60C6C8}
    \definecolor{ansi-cyan-intense}{HTML}{258F8F}
    \definecolor{ansi-white}{HTML}{C5C1B4}
    \definecolor{ansi-white-intense}{HTML}{A1A6B2}

    % commands and environments needed by pandoc snippets
    % extracted from the output of `pandoc -s`
    \providecommand{\tightlist}{%
      \setlength{\itemsep}{0pt}\setlength{\parskip}{0pt}}
    \DefineVerbatimEnvironment{Highlighting}{Verbatim}{commandchars=\\\{\}}
    % Add ',fontsize=\small' for more characters per line
    \newenvironment{Shaded}{}{}
    \newcommand{\KeywordTok}[1]{\textcolor[rgb]{0.00,0.44,0.13}{\textbf{{#1}}}}
    \newcommand{\DataTypeTok}[1]{\textcolor[rgb]{0.56,0.13,0.00}{{#1}}}
    \newcommand{\DecValTok}[1]{\textcolor[rgb]{0.25,0.63,0.44}{{#1}}}
    \newcommand{\BaseNTok}[1]{\textcolor[rgb]{0.25,0.63,0.44}{{#1}}}
    \newcommand{\FloatTok}[1]{\textcolor[rgb]{0.25,0.63,0.44}{{#1}}}
    \newcommand{\CharTok}[1]{\textcolor[rgb]{0.25,0.44,0.63}{{#1}}}
    \newcommand{\StringTok}[1]{\textcolor[rgb]{0.25,0.44,0.63}{{#1}}}
    \newcommand{\CommentTok}[1]{\textcolor[rgb]{0.38,0.63,0.69}{\textit{{#1}}}}
    \newcommand{\OtherTok}[1]{\textcolor[rgb]{0.00,0.44,0.13}{{#1}}}
    \newcommand{\AlertTok}[1]{\textcolor[rgb]{1.00,0.00,0.00}{\textbf{{#1}}}}
    \newcommand{\FunctionTok}[1]{\textcolor[rgb]{0.02,0.16,0.49}{{#1}}}
    \newcommand{\RegionMarkerTok}[1]{{#1}}
    \newcommand{\ErrorTok}[1]{\textcolor[rgb]{1.00,0.00,0.00}{\textbf{{#1}}}}
    \newcommand{\NormalTok}[1]{{#1}}
    
    % Additional commands for more recent versions of Pandoc
    \newcommand{\ConstantTok}[1]{\textcolor[rgb]{0.53,0.00,0.00}{{#1}}}
    \newcommand{\SpecialCharTok}[1]{\textcolor[rgb]{0.25,0.44,0.63}{{#1}}}
    \newcommand{\VerbatimStringTok}[1]{\textcolor[rgb]{0.25,0.44,0.63}{{#1}}}
    \newcommand{\SpecialStringTok}[1]{\textcolor[rgb]{0.73,0.40,0.53}{{#1}}}
    \newcommand{\ImportTok}[1]{{#1}}
    \newcommand{\DocumentationTok}[1]{\textcolor[rgb]{0.73,0.13,0.13}{\textit{{#1}}}}
    \newcommand{\AnnotationTok}[1]{\textcolor[rgb]{0.38,0.63,0.69}{\textbf{\textit{{#1}}}}}
    \newcommand{\CommentVarTok}[1]{\textcolor[rgb]{0.38,0.63,0.69}{\textbf{\textit{{#1}}}}}
    \newcommand{\VariableTok}[1]{\textcolor[rgb]{0.10,0.09,0.49}{{#1}}}
    \newcommand{\ControlFlowTok}[1]{\textcolor[rgb]{0.00,0.44,0.13}{\textbf{{#1}}}}
    \newcommand{\OperatorTok}[1]{\textcolor[rgb]{0.40,0.40,0.40}{{#1}}}
    \newcommand{\BuiltInTok}[1]{{#1}}
    \newcommand{\ExtensionTok}[1]{{#1}}
    \newcommand{\PreprocessorTok}[1]{\textcolor[rgb]{0.74,0.48,0.00}{{#1}}}
    \newcommand{\AttributeTok}[1]{\textcolor[rgb]{0.49,0.56,0.16}{{#1}}}
    \newcommand{\InformationTok}[1]{\textcolor[rgb]{0.38,0.63,0.69}{\textbf{\textit{{#1}}}}}
    \newcommand{\WarningTok}[1]{\textcolor[rgb]{0.38,0.63,0.69}{\textbf{\textit{{#1}}}}}
    
    
    % Define a nice break command that doesn't care if a line doesn't already
    % exist.
    \def\br{\hspace*{\fill} \\* }
    % Math Jax compatability definitions
    \def\gt{>}
    \def\lt{<}
    % Document parameters
    \title{Trabajo Final Clasificaci?n}
    
    
    

    % Pygments definitions
    
\makeatletter
\def\PY@reset{\let\PY@it=\relax \let\PY@bf=\relax%
    \let\PY@ul=\relax \let\PY@tc=\relax%
    \let\PY@bc=\relax \let\PY@ff=\relax}
\def\PY@tok#1{\csname PY@tok@#1\endcsname}
\def\PY@toks#1+{\ifx\relax#1\empty\else%
    \PY@tok{#1}\expandafter\PY@toks\fi}
\def\PY@do#1{\PY@bc{\PY@tc{\PY@ul{%
    \PY@it{\PY@bf{\PY@ff{#1}}}}}}}
\def\PY#1#2{\PY@reset\PY@toks#1+\relax+\PY@do{#2}}

\expandafter\def\csname PY@tok@w\endcsname{\def\PY@tc##1{\textcolor[rgb]{0.73,0.73,0.73}{##1}}}
\expandafter\def\csname PY@tok@c\endcsname{\let\PY@it=\textit\def\PY@tc##1{\textcolor[rgb]{0.25,0.50,0.50}{##1}}}
\expandafter\def\csname PY@tok@cp\endcsname{\def\PY@tc##1{\textcolor[rgb]{0.74,0.48,0.00}{##1}}}
\expandafter\def\csname PY@tok@k\endcsname{\let\PY@bf=\textbf\def\PY@tc##1{\textcolor[rgb]{0.00,0.50,0.00}{##1}}}
\expandafter\def\csname PY@tok@kp\endcsname{\def\PY@tc##1{\textcolor[rgb]{0.00,0.50,0.00}{##1}}}
\expandafter\def\csname PY@tok@kt\endcsname{\def\PY@tc##1{\textcolor[rgb]{0.69,0.00,0.25}{##1}}}
\expandafter\def\csname PY@tok@o\endcsname{\def\PY@tc##1{\textcolor[rgb]{0.40,0.40,0.40}{##1}}}
\expandafter\def\csname PY@tok@ow\endcsname{\let\PY@bf=\textbf\def\PY@tc##1{\textcolor[rgb]{0.67,0.13,1.00}{##1}}}
\expandafter\def\csname PY@tok@nb\endcsname{\def\PY@tc##1{\textcolor[rgb]{0.00,0.50,0.00}{##1}}}
\expandafter\def\csname PY@tok@nf\endcsname{\def\PY@tc##1{\textcolor[rgb]{0.00,0.00,1.00}{##1}}}
\expandafter\def\csname PY@tok@nc\endcsname{\let\PY@bf=\textbf\def\PY@tc##1{\textcolor[rgb]{0.00,0.00,1.00}{##1}}}
\expandafter\def\csname PY@tok@nn\endcsname{\let\PY@bf=\textbf\def\PY@tc##1{\textcolor[rgb]{0.00,0.00,1.00}{##1}}}
\expandafter\def\csname PY@tok@ne\endcsname{\let\PY@bf=\textbf\def\PY@tc##1{\textcolor[rgb]{0.82,0.25,0.23}{##1}}}
\expandafter\def\csname PY@tok@nv\endcsname{\def\PY@tc##1{\textcolor[rgb]{0.10,0.09,0.49}{##1}}}
\expandafter\def\csname PY@tok@no\endcsname{\def\PY@tc##1{\textcolor[rgb]{0.53,0.00,0.00}{##1}}}
\expandafter\def\csname PY@tok@nl\endcsname{\def\PY@tc##1{\textcolor[rgb]{0.63,0.63,0.00}{##1}}}
\expandafter\def\csname PY@tok@ni\endcsname{\let\PY@bf=\textbf\def\PY@tc##1{\textcolor[rgb]{0.60,0.60,0.60}{##1}}}
\expandafter\def\csname PY@tok@na\endcsname{\def\PY@tc##1{\textcolor[rgb]{0.49,0.56,0.16}{##1}}}
\expandafter\def\csname PY@tok@nt\endcsname{\let\PY@bf=\textbf\def\PY@tc##1{\textcolor[rgb]{0.00,0.50,0.00}{##1}}}
\expandafter\def\csname PY@tok@nd\endcsname{\def\PY@tc##1{\textcolor[rgb]{0.67,0.13,1.00}{##1}}}
\expandafter\def\csname PY@tok@s\endcsname{\def\PY@tc##1{\textcolor[rgb]{0.73,0.13,0.13}{##1}}}
\expandafter\def\csname PY@tok@sd\endcsname{\let\PY@it=\textit\def\PY@tc##1{\textcolor[rgb]{0.73,0.13,0.13}{##1}}}
\expandafter\def\csname PY@tok@si\endcsname{\let\PY@bf=\textbf\def\PY@tc##1{\textcolor[rgb]{0.73,0.40,0.53}{##1}}}
\expandafter\def\csname PY@tok@se\endcsname{\let\PY@bf=\textbf\def\PY@tc##1{\textcolor[rgb]{0.73,0.40,0.13}{##1}}}
\expandafter\def\csname PY@tok@sr\endcsname{\def\PY@tc##1{\textcolor[rgb]{0.73,0.40,0.53}{##1}}}
\expandafter\def\csname PY@tok@ss\endcsname{\def\PY@tc##1{\textcolor[rgb]{0.10,0.09,0.49}{##1}}}
\expandafter\def\csname PY@tok@sx\endcsname{\def\PY@tc##1{\textcolor[rgb]{0.00,0.50,0.00}{##1}}}
\expandafter\def\csname PY@tok@m\endcsname{\def\PY@tc##1{\textcolor[rgb]{0.40,0.40,0.40}{##1}}}
\expandafter\def\csname PY@tok@gh\endcsname{\let\PY@bf=\textbf\def\PY@tc##1{\textcolor[rgb]{0.00,0.00,0.50}{##1}}}
\expandafter\def\csname PY@tok@gu\endcsname{\let\PY@bf=\textbf\def\PY@tc##1{\textcolor[rgb]{0.50,0.00,0.50}{##1}}}
\expandafter\def\csname PY@tok@gd\endcsname{\def\PY@tc##1{\textcolor[rgb]{0.63,0.00,0.00}{##1}}}
\expandafter\def\csname PY@tok@gi\endcsname{\def\PY@tc##1{\textcolor[rgb]{0.00,0.63,0.00}{##1}}}
\expandafter\def\csname PY@tok@gr\endcsname{\def\PY@tc##1{\textcolor[rgb]{1.00,0.00,0.00}{##1}}}
\expandafter\def\csname PY@tok@ge\endcsname{\let\PY@it=\textit}
\expandafter\def\csname PY@tok@gs\endcsname{\let\PY@bf=\textbf}
\expandafter\def\csname PY@tok@gp\endcsname{\let\PY@bf=\textbf\def\PY@tc##1{\textcolor[rgb]{0.00,0.00,0.50}{##1}}}
\expandafter\def\csname PY@tok@go\endcsname{\def\PY@tc##1{\textcolor[rgb]{0.53,0.53,0.53}{##1}}}
\expandafter\def\csname PY@tok@gt\endcsname{\def\PY@tc##1{\textcolor[rgb]{0.00,0.27,0.87}{##1}}}
\expandafter\def\csname PY@tok@err\endcsname{\def\PY@bc##1{\setlength{\fboxsep}{0pt}\fcolorbox[rgb]{1.00,0.00,0.00}{1,1,1}{\strut ##1}}}
\expandafter\def\csname PY@tok@kc\endcsname{\let\PY@bf=\textbf\def\PY@tc##1{\textcolor[rgb]{0.00,0.50,0.00}{##1}}}
\expandafter\def\csname PY@tok@kd\endcsname{\let\PY@bf=\textbf\def\PY@tc##1{\textcolor[rgb]{0.00,0.50,0.00}{##1}}}
\expandafter\def\csname PY@tok@kn\endcsname{\let\PY@bf=\textbf\def\PY@tc##1{\textcolor[rgb]{0.00,0.50,0.00}{##1}}}
\expandafter\def\csname PY@tok@kr\endcsname{\let\PY@bf=\textbf\def\PY@tc##1{\textcolor[rgb]{0.00,0.50,0.00}{##1}}}
\expandafter\def\csname PY@tok@bp\endcsname{\def\PY@tc##1{\textcolor[rgb]{0.00,0.50,0.00}{##1}}}
\expandafter\def\csname PY@tok@fm\endcsname{\def\PY@tc##1{\textcolor[rgb]{0.00,0.00,1.00}{##1}}}
\expandafter\def\csname PY@tok@vc\endcsname{\def\PY@tc##1{\textcolor[rgb]{0.10,0.09,0.49}{##1}}}
\expandafter\def\csname PY@tok@vg\endcsname{\def\PY@tc##1{\textcolor[rgb]{0.10,0.09,0.49}{##1}}}
\expandafter\def\csname PY@tok@vi\endcsname{\def\PY@tc##1{\textcolor[rgb]{0.10,0.09,0.49}{##1}}}
\expandafter\def\csname PY@tok@vm\endcsname{\def\PY@tc##1{\textcolor[rgb]{0.10,0.09,0.49}{##1}}}
\expandafter\def\csname PY@tok@sa\endcsname{\def\PY@tc##1{\textcolor[rgb]{0.73,0.13,0.13}{##1}}}
\expandafter\def\csname PY@tok@sb\endcsname{\def\PY@tc##1{\textcolor[rgb]{0.73,0.13,0.13}{##1}}}
\expandafter\def\csname PY@tok@sc\endcsname{\def\PY@tc##1{\textcolor[rgb]{0.73,0.13,0.13}{##1}}}
\expandafter\def\csname PY@tok@dl\endcsname{\def\PY@tc##1{\textcolor[rgb]{0.73,0.13,0.13}{##1}}}
\expandafter\def\csname PY@tok@s2\endcsname{\def\PY@tc##1{\textcolor[rgb]{0.73,0.13,0.13}{##1}}}
\expandafter\def\csname PY@tok@sh\endcsname{\def\PY@tc##1{\textcolor[rgb]{0.73,0.13,0.13}{##1}}}
\expandafter\def\csname PY@tok@s1\endcsname{\def\PY@tc##1{\textcolor[rgb]{0.73,0.13,0.13}{##1}}}
\expandafter\def\csname PY@tok@mb\endcsname{\def\PY@tc##1{\textcolor[rgb]{0.40,0.40,0.40}{##1}}}
\expandafter\def\csname PY@tok@mf\endcsname{\def\PY@tc##1{\textcolor[rgb]{0.40,0.40,0.40}{##1}}}
\expandafter\def\csname PY@tok@mh\endcsname{\def\PY@tc##1{\textcolor[rgb]{0.40,0.40,0.40}{##1}}}
\expandafter\def\csname PY@tok@mi\endcsname{\def\PY@tc##1{\textcolor[rgb]{0.40,0.40,0.40}{##1}}}
\expandafter\def\csname PY@tok@il\endcsname{\def\PY@tc##1{\textcolor[rgb]{0.40,0.40,0.40}{##1}}}
\expandafter\def\csname PY@tok@mo\endcsname{\def\PY@tc##1{\textcolor[rgb]{0.40,0.40,0.40}{##1}}}
\expandafter\def\csname PY@tok@ch\endcsname{\let\PY@it=\textit\def\PY@tc##1{\textcolor[rgb]{0.25,0.50,0.50}{##1}}}
\expandafter\def\csname PY@tok@cm\endcsname{\let\PY@it=\textit\def\PY@tc##1{\textcolor[rgb]{0.25,0.50,0.50}{##1}}}
\expandafter\def\csname PY@tok@cpf\endcsname{\let\PY@it=\textit\def\PY@tc##1{\textcolor[rgb]{0.25,0.50,0.50}{##1}}}
\expandafter\def\csname PY@tok@c1\endcsname{\let\PY@it=\textit\def\PY@tc##1{\textcolor[rgb]{0.25,0.50,0.50}{##1}}}
\expandafter\def\csname PY@tok@cs\endcsname{\let\PY@it=\textit\def\PY@tc##1{\textcolor[rgb]{0.25,0.50,0.50}{##1}}}

\def\PYZbs{\char`\\}
\def\PYZus{\char`\_}
\def\PYZob{\char`\{}
\def\PYZcb{\char`\}}
\def\PYZca{\char`\^}
\def\PYZam{\char`\&}
\def\PYZlt{\char`\<}
\def\PYZgt{\char`\>}
\def\PYZsh{\char`\#}
\def\PYZpc{\char`\%}
\def\PYZdl{\char`\$}
\def\PYZhy{\char`\-}
\def\PYZsq{\char`\'}
\def\PYZdq{\char`\"}
\def\PYZti{\char`\~}
% for compatibility with earlier versions
\def\PYZat{@}
\def\PYZlb{[}
\def\PYZrb{]}
\makeatother


    % Exact colors from NB
    \definecolor{incolor}{rgb}{0.0, 0.0, 0.5}
    \definecolor{outcolor}{rgb}{0.545, 0.0, 0.0}



    
    % Prevent overflowing lines due to hard-to-break entities
    \sloppy 
    % Setup hyperref package
    \hypersetup{
      breaklinks=true,  % so long urls are correctly broken across lines
      colorlinks=true,
      urlcolor=urlcolor,
      linkcolor=linkcolor,
      citecolor=citecolor,
      }
    % Slightly bigger margins than the latex defaults
    
    \geometry{verbose,tmargin=1in,bmargin=1in,lmargin=1in,rmargin=1in}
    
    

    \begin{document}
    
    
    \maketitle
    
    

    
    \hypertarget{clasificaciuxf3n-sobre-el-dataset-hayes-roth}{%
\subsection{Clasificación sobre el dataset
hayes-roth}\label{clasificaciuxf3n-sobre-el-dataset-hayes-roth}}

Alberto Armijo Ruiz

    \hypertarget{informaciuxf3n-sobre-el-dataset}{%
\subsubsection{Información sobre el
dataset}\label{informaciuxf3n-sobre-el-dataset}}

El dataset \emph{hayes-roth} se trata de un dataset artificial utilizado
para estudiar el comportamiento de clasificadores , el atributo
\emph{Hobby} está generado de forma aleatoria y se utiliza para
introducir ruido dentro del dataset. Este dataset cuenta con cuatro
variables predictoras: \emph{Hobby}, \emph{Age}, \emph{EducationalLevel}
y \emph{MaritalStatus}; también contamos con una variable, llamada
\emph{Class} que representa la clase en la que se encuentra cada dato.

Cada una de las variables predictoras se encuentran con valores enteros
del 1 al 4, menos la variable \emph{Hobby} que va del 1 al 3. La
variable que tenemos que predecir cuenta con tres clases (1,2 y 3).

Este dataset cuenta con 160 instancias y no cuenta con datos pérdidos.
Se puede encontrar información sobre este dataset en
\url{https://sci2s.ugr.es/keel/dataset.php?cod=186} .

    \hypertarget{anuxe1lisis-exploratorio-de-los-datos}{%
\subsection{Análisis exploratorio de los
datos}\label{anuxe1lisis-exploratorio-de-los-datos}}

Para leer los datos, deberemos que leer el archivo llamado
\emph{hayes-roth.dat} que se encuentra dentro de la carpeta
\emph{hayes-roth}. Tras esto, deberemos añadirle los nombres a las
variables.

    \begin{Verbatim}[commandchars=\\\{\}]
{\color{incolor}In [{\color{incolor}8}]:} \PY{c+c1}{\PYZsh{} Leemos el dataset}
        \PY{n}{hayesroth} \PY{o}{=} \PY{n+nf}{read.csv}\PY{p}{(}\PY{l+s}{\PYZsq{}}\PY{l+s}{hayes\PYZhy{}roth//hayes\PYZhy{}roth.dat\PYZsq{}}\PY{p}{,} \PY{n}{comment.char} \PY{o}{=} \PY{l+s}{\PYZsq{}}\PY{l+s}{@\PYZsq{}}\PY{p}{,} \PY{n}{header} \PY{o}{=} \PY{k+kc}{FALSE}\PY{p}{)}
        \PY{n}{nombres} \PY{o}{=} \PY{n+nf}{c}\PY{p}{(}\PY{l+s}{\PYZsq{}}\PY{l+s}{Hobby\PYZsq{}}\PY{p}{,} \PY{l+s}{\PYZsq{}}\PY{l+s}{Age\PYZsq{}}\PY{p}{,} \PY{l+s}{\PYZsq{}}\PY{l+s}{EducationalLevel\PYZsq{}}\PY{p}{,} \PY{l+s}{\PYZsq{}}\PY{l+s}{MaritalStatus\PYZsq{}}\PY{p}{,}\PY{l+s}{\PYZsq{}}\PY{l+s}{Class\PYZsq{}}\PY{p}{)}
        \PY{n+nf}{colnames}\PY{p}{(}\PY{n}{hayesroth}\PY{p}{)} \PY{o}{=} \PY{n}{nombres}
        
        \PY{n+nf}{head}\PY{p}{(}\PY{n}{hayesroth}\PY{p}{)}
        \PY{n+nf}{str}\PY{p}{(}\PY{n}{hayesroth}\PY{p}{)}
\end{Verbatim}


    \begin{tabular}{r|lllll}
 Hobby & Age & EducationalLevel & MaritalStatus & Class\\
\hline
	 2 & 1 & 1 & 2 & 1\\
	 2 & 1 & 3 & 2 & 2\\
	 3 & 1 & 4 & 1 & 3\\
	 2 & 4 & 2 & 2 & 3\\
	 1 & 1 & 3 & 4 & 3\\
	 1 & 1 & 3 & 2 & 2\\
\end{tabular}


    
    \begin{Verbatim}[commandchars=\\\{\}]
'data.frame':	160 obs. of  5 variables:
 \$ Hobby           : int  2 2 3 2 1 1 3 3 2 1 {\ldots}
 \$ Age             : int  1 1 1 4 1 1 1 4 2 2 {\ldots}
 \$ EducationalLevel: int  1 3 4 2 3 3 3 2 1 1 {\ldots}
 \$ MaritalStatus   : int  2 2 1 2 4 2 2 4 1 1 {\ldots}
 \$ Class           : int  1 2 3 3 3 2 2 3 1 1 {\ldots}

    \end{Verbatim}

    \hypertarget{cuxe1lculo-de-medias-y-desviaciones}{%
\subsubsection{Cálculo de medias y
desviaciones}\label{cuxe1lculo-de-medias-y-desviaciones}}

    Ahora calcularemos la media y desviación estandar de cada uno de los
atributos.

    \begin{Verbatim}[commandchars=\\\{\}]
{\color{incolor}In [{\color{incolor}9}]:} \PY{c+c1}{\PYZsh{} Calculamos la media y la desviación.}
        \PY{n}{medias} \PY{o}{=} \PY{n+nf}{sapply}\PY{p}{(}\PY{n}{hayesroth}\PY{p}{,} \PY{n}{mean}\PY{p}{)}
        \PY{n}{desviaciones} \PY{o}{=} \PY{n+nf}{sapply}\PY{p}{(}\PY{n}{hayesroth}\PY{p}{,} \PY{n}{sd}\PY{p}{)}
        
        \PY{c+c1}{\PYZsh{} Creamos una tabla y mostramos los resultados.}
        \PY{n}{medydesv} \PY{o}{=} \PY{n+nf}{cbind}\PY{p}{(}\PY{n}{medias}\PY{p}{,} \PY{n}{desviaciones}\PY{p}{)}
        \PY{n}{medydesv}
\end{Verbatim}


    \begin{tabular}{r|ll}
  & medias & desviaciones\\
\hline
	Hobby & 1.8250    & 0.8359080\\
	Age & 1.9750    & 0.9380161\\
	EducationalLevel & 1.9750    & 0.9380161\\
	MaritalStatus & 1.9750    & 0.9380161\\
	Class & 1.7875    & 0.7472171\\
\end{tabular}


    
    Por lo que podemos ver, las medias y desviaciones tiene bastante
sentido, por lo que parece que los datos están bien distribuido, de
todas formas, pintaremos las distribuciones de las variables.

    \hypertarget{representaciuxf3n-de-los-datos}{%
\subsubsection{Representación de los
datos}\label{representaciuxf3n-de-los-datos}}

    \begin{Verbatim}[commandchars=\\\{\}]
{\color{incolor}In [{\color{incolor}10}]:} \PY{c+c1}{\PYZsh{} cargamos la librería.}
         \PY{n+nf}{library}\PY{p}{(}\PY{n}{ggplot2}\PY{p}{)}
\end{Verbatim}


    \begin{Verbatim}[commandchars=\\\{\}]
{\color{incolor}In [{\color{incolor}11}]:} \PY{c+c1}{\PYZsh{} Pintamos un histograma por cada variable. Dibujaremos la media y la mediana también.}
         \PY{n+nf}{ggplot}\PY{p}{(}\PY{n}{hayesroth}\PY{p}{,} \PY{n+nf}{aes}\PY{p}{(}\PY{n}{x}\PY{o}{=}\PY{n}{Hobby}\PY{p}{)}\PY{p}{)}\PY{o}{+}
             \PY{n+nf}{geom\PYZus{}histogram}\PY{p}{(}\PY{n}{bins} \PY{o}{=} \PY{n+nf}{length}\PY{p}{(}\PY{n+nf}{unique}\PY{p}{(}\PY{n}{hayesroth}\PY{o}{\PYZdl{}}\PY{n}{Hobby}\PY{p}{)}\PY{p}{)}\PY{p}{,} \PY{n}{fill}\PY{o}{=}\PY{l+s}{\PYZdq{}}\PY{l+s}{lightblue\PYZdq{}}\PY{p}{,} \PY{n}{color}\PY{o}{=}\PY{l+s}{\PYZdq{}}\PY{l+s}{black\PYZdq{}}\PY{p}{)}\PY{o}{+}
             \PY{n+nf}{geom\PYZus{}vline}\PY{p}{(}\PY{n+nf}{aes}\PY{p}{(}\PY{n}{xintercept} \PY{o}{=} \PY{n+nf}{mean}\PY{p}{(}\PY{n}{Hobby}\PY{p}{)}\PY{p}{,}\PY{n}{color} \PY{o}{=} \PY{l+s}{\PYZdq{}}\PY{l+s}{mean\PYZdq{}}\PY{p}{)}\PY{p}{,} 
                      \PY{n}{linetype} \PY{o}{=} \PY{l+s}{\PYZdq{}}\PY{l+s}{dashed\PYZdq{}}\PY{p}{,} \PY{n}{size} \PY{o}{=} \PY{l+m}{0.6}\PY{p}{)} \PY{o}{+}
             \PY{n+nf}{geom\PYZus{}vline}\PY{p}{(}\PY{n+nf}{aes}\PY{p}{(}\PY{n}{xintercept} \PY{o}{=} \PY{n+nf}{median}\PY{p}{(}\PY{n}{Hobby}\PY{p}{)}\PY{p}{,}\PY{n}{color}\PY{o}{=}\PY{l+s}{\PYZdq{}}\PY{l+s}{median\PYZdq{}}\PY{p}{)}\PY{p}{,}
                           \PY{n}{linetype} \PY{o}{=} \PY{l+s}{\PYZdq{}}\PY{l+s}{dashed\PYZdq{}}\PY{p}{,} \PY{n}{size} \PY{o}{=} \PY{l+m}{0.6} \PY{p}{)} \PY{o}{+}
             \PY{n+nf}{scale\PYZus{}color\PYZus{}manual}\PY{p}{(}\PY{n}{name} \PY{o}{=} \PY{l+s}{\PYZdq{}}\PY{l+s}{statistics\PYZdq{}}\PY{p}{,} \PY{n}{values} \PY{o}{=} \PY{n+nf}{c}\PY{p}{(}\PY{n}{mean} \PY{o}{=} \PY{l+s}{\PYZdq{}}\PY{l+s}{\PYZsh{}FC4E07\PYZdq{}}\PY{p}{,} \PY{n}{median} \PY{o}{=} \PY{l+s}{\PYZdq{}}\PY{l+s}{blue\PYZdq{}}\PY{p}{)}\PY{p}{)}
\end{Verbatim}


    
    
    \begin{center}
    \adjustimage{max size={0.9\linewidth}{0.9\paperheight}}{output_10_1.png}
    \end{center}
    { \hspace*{\fill} \\}
    
    \begin{Verbatim}[commandchars=\\\{\}]
{\color{incolor}In [{\color{incolor}12}]:} \PY{c+c1}{\PYZsh{} Pintamos un histograma por cada variable. Dibujaremos la media y la mediana también.}
         \PY{n+nf}{ggplot}\PY{p}{(}\PY{n}{hayesroth}\PY{p}{,} \PY{n+nf}{aes}\PY{p}{(}\PY{n}{x}\PY{o}{=}\PY{n}{Age}\PY{p}{)}\PY{p}{)}\PY{o}{+}
             \PY{n+nf}{geom\PYZus{}histogram}\PY{p}{(}\PY{n}{bins} \PY{o}{=} \PY{n+nf}{length}\PY{p}{(}\PY{n+nf}{unique}\PY{p}{(}\PY{n}{hayesroth}\PY{o}{\PYZdl{}}\PY{n}{Age}\PY{p}{)}\PY{p}{)}\PY{p}{,} \PY{n}{fill}\PY{o}{=}\PY{l+s}{\PYZdq{}}\PY{l+s}{lightblue\PYZdq{}}\PY{p}{,} \PY{n}{color}\PY{o}{=}\PY{l+s}{\PYZdq{}}\PY{l+s}{black\PYZdq{}}\PY{p}{)}\PY{o}{+}
             \PY{n+nf}{geom\PYZus{}vline}\PY{p}{(}\PY{n+nf}{aes}\PY{p}{(}\PY{n}{xintercept} \PY{o}{=} \PY{n+nf}{mean}\PY{p}{(}\PY{n}{Age}\PY{p}{)}\PY{p}{,}\PY{n}{color} \PY{o}{=} \PY{l+s}{\PYZdq{}}\PY{l+s}{mean\PYZdq{}}\PY{p}{)}\PY{p}{,} 
                      \PY{n}{linetype} \PY{o}{=} \PY{l+s}{\PYZdq{}}\PY{l+s}{dashed\PYZdq{}}\PY{p}{,} \PY{n}{size} \PY{o}{=} \PY{l+m}{0.6}\PY{p}{)} \PY{o}{+}
             \PY{n+nf}{geom\PYZus{}vline}\PY{p}{(}\PY{n+nf}{aes}\PY{p}{(}\PY{n}{xintercept} \PY{o}{=} \PY{n+nf}{median}\PY{p}{(}\PY{n}{Age}\PY{p}{)}\PY{p}{,}\PY{n}{color}\PY{o}{=}\PY{l+s}{\PYZdq{}}\PY{l+s}{median\PYZdq{}}\PY{p}{)}\PY{p}{,}
                           \PY{n}{linetype} \PY{o}{=} \PY{l+s}{\PYZdq{}}\PY{l+s}{dashed\PYZdq{}}\PY{p}{,} \PY{n}{size} \PY{o}{=} \PY{l+m}{0.6} \PY{p}{)} \PY{o}{+}
             \PY{n+nf}{scale\PYZus{}color\PYZus{}manual}\PY{p}{(}\PY{n}{name} \PY{o}{=} \PY{l+s}{\PYZdq{}}\PY{l+s}{statistics\PYZdq{}}\PY{p}{,} \PY{n}{values} \PY{o}{=} \PY{n+nf}{c}\PY{p}{(}\PY{n}{mean} \PY{o}{=} \PY{l+s}{\PYZdq{}}\PY{l+s}{\PYZsh{}FC4E07\PYZdq{}}\PY{p}{,} \PY{n}{median} \PY{o}{=} \PY{l+s}{\PYZdq{}}\PY{l+s}{blue\PYZdq{}}\PY{p}{)}\PY{p}{)}
\end{Verbatim}


    
    
    \begin{center}
    \adjustimage{max size={0.9\linewidth}{0.9\paperheight}}{output_11_1.png}
    \end{center}
    { \hspace*{\fill} \\}
    
    \begin{Verbatim}[commandchars=\\\{\}]
{\color{incolor}In [{\color{incolor}13}]:} \PY{c+c1}{\PYZsh{} Pintamos un histograma por cada variable. Dibujaremos la media y la mediana también.}
         \PY{n+nf}{ggplot}\PY{p}{(}\PY{n}{hayesroth}\PY{p}{,} \PY{n+nf}{aes}\PY{p}{(}\PY{n}{x}\PY{o}{=}\PY{n}{EducationalLevel}\PY{p}{)}\PY{p}{)}\PY{o}{+}
             \PY{n+nf}{geom\PYZus{}histogram}\PY{p}{(}\PY{n}{bins} \PY{o}{=} \PY{n+nf}{length}\PY{p}{(}\PY{n+nf}{unique}\PY{p}{(}\PY{n}{hayesroth}\PY{o}{\PYZdl{}}\PY{n}{EducationalLevel}\PY{p}{)}\PY{p}{)}\PY{p}{,} \PY{n}{fill}\PY{o}{=}\PY{l+s}{\PYZdq{}}\PY{l+s}{lightblue\PYZdq{}}\PY{p}{,} \PY{n}{color}\PY{o}{=}\PY{l+s}{\PYZdq{}}\PY{l+s}{black\PYZdq{}}\PY{p}{)}\PY{o}{+}
             \PY{n+nf}{geom\PYZus{}vline}\PY{p}{(}\PY{n+nf}{aes}\PY{p}{(}\PY{n}{xintercept} \PY{o}{=} \PY{n+nf}{mean}\PY{p}{(}\PY{n}{EducationalLevel}\PY{p}{)}\PY{p}{,}\PY{n}{color} \PY{o}{=} \PY{l+s}{\PYZdq{}}\PY{l+s}{mean\PYZdq{}}\PY{p}{)}\PY{p}{,} 
                      \PY{n}{linetype} \PY{o}{=} \PY{l+s}{\PYZdq{}}\PY{l+s}{dashed\PYZdq{}}\PY{p}{,} \PY{n}{size} \PY{o}{=} \PY{l+m}{0.6}\PY{p}{)} \PY{o}{+}
             \PY{n+nf}{geom\PYZus{}vline}\PY{p}{(}\PY{n+nf}{aes}\PY{p}{(}\PY{n}{xintercept} \PY{o}{=} \PY{n+nf}{median}\PY{p}{(}\PY{n}{EducationalLevel}\PY{p}{)}\PY{p}{,}\PY{n}{color}\PY{o}{=}\PY{l+s}{\PYZdq{}}\PY{l+s}{median\PYZdq{}}\PY{p}{)}\PY{p}{,}
                           \PY{n}{linetype} \PY{o}{=} \PY{l+s}{\PYZdq{}}\PY{l+s}{dashed\PYZdq{}}\PY{p}{,} \PY{n}{size} \PY{o}{=} \PY{l+m}{0.6} \PY{p}{)} \PY{o}{+}
             \PY{n+nf}{scale\PYZus{}color\PYZus{}manual}\PY{p}{(}\PY{n}{name} \PY{o}{=} \PY{l+s}{\PYZdq{}}\PY{l+s}{statistics\PYZdq{}}\PY{p}{,} \PY{n}{values} \PY{o}{=} \PY{n+nf}{c}\PY{p}{(}\PY{n}{mean} \PY{o}{=} \PY{l+s}{\PYZdq{}}\PY{l+s}{\PYZsh{}FC4E07\PYZdq{}}\PY{p}{,} \PY{n}{median} \PY{o}{=} \PY{l+s}{\PYZdq{}}\PY{l+s}{blue\PYZdq{}}\PY{p}{)}\PY{p}{)}
\end{Verbatim}


    
    
    \begin{center}
    \adjustimage{max size={0.9\linewidth}{0.9\paperheight}}{output_12_1.png}
    \end{center}
    { \hspace*{\fill} \\}
    
    \begin{Verbatim}[commandchars=\\\{\}]
{\color{incolor}In [{\color{incolor}14}]:} \PY{c+c1}{\PYZsh{} Pintamos un histograma por cada variable. Dibujaremos la media y la mediana también.}
         \PY{n+nf}{ggplot}\PY{p}{(}\PY{n}{hayesroth}\PY{p}{,} \PY{n+nf}{aes}\PY{p}{(}\PY{n}{x}\PY{o}{=}\PY{n}{MaritalStatus}\PY{p}{)}\PY{p}{)}\PY{o}{+}
             \PY{n+nf}{geom\PYZus{}histogram}\PY{p}{(}\PY{n}{bins} \PY{o}{=} \PY{n+nf}{length}\PY{p}{(}\PY{n+nf}{unique}\PY{p}{(}\PY{n}{hayesroth}\PY{o}{\PYZdl{}}\PY{n}{MaritalStatus}\PY{p}{)}\PY{p}{)}\PY{p}{,} \PY{n}{fill}\PY{o}{=}\PY{l+s}{\PYZdq{}}\PY{l+s}{lightblue\PYZdq{}}\PY{p}{,} \PY{n}{color}\PY{o}{=}\PY{l+s}{\PYZdq{}}\PY{l+s}{black\PYZdq{}}\PY{p}{)}\PY{o}{+}
             \PY{n+nf}{geom\PYZus{}vline}\PY{p}{(}\PY{n+nf}{aes}\PY{p}{(}\PY{n}{xintercept} \PY{o}{=} \PY{n+nf}{mean}\PY{p}{(}\PY{n}{MaritalStatus}\PY{p}{)}\PY{p}{,}\PY{n}{color} \PY{o}{=} \PY{l+s}{\PYZdq{}}\PY{l+s}{mean\PYZdq{}}\PY{p}{)}\PY{p}{,} 
                      \PY{n}{linetype} \PY{o}{=} \PY{l+s}{\PYZdq{}}\PY{l+s}{dashed\PYZdq{}}\PY{p}{,} \PY{n}{size} \PY{o}{=} \PY{l+m}{0.6}\PY{p}{)} \PY{o}{+}
             \PY{n+nf}{geom\PYZus{}vline}\PY{p}{(}\PY{n+nf}{aes}\PY{p}{(}\PY{n}{xintercept} \PY{o}{=} \PY{n+nf}{median}\PY{p}{(}\PY{n}{MaritalStatus}\PY{p}{)}\PY{p}{,}\PY{n}{color}\PY{o}{=}\PY{l+s}{\PYZdq{}}\PY{l+s}{median\PYZdq{}}\PY{p}{)}\PY{p}{,}
                           \PY{n}{linetype} \PY{o}{=} \PY{l+s}{\PYZdq{}}\PY{l+s}{dashed\PYZdq{}}\PY{p}{,} \PY{n}{size} \PY{o}{=} \PY{l+m}{0.6} \PY{p}{)} \PY{o}{+}
             \PY{n+nf}{scale\PYZus{}color\PYZus{}manual}\PY{p}{(}\PY{n}{name} \PY{o}{=} \PY{l+s}{\PYZdq{}}\PY{l+s}{statistics\PYZdq{}}\PY{p}{,} \PY{n}{values} \PY{o}{=} \PY{n+nf}{c}\PY{p}{(}\PY{n}{mean} \PY{o}{=} \PY{l+s}{\PYZdq{}}\PY{l+s}{\PYZsh{}FC4E07\PYZdq{}}\PY{p}{,} \PY{n}{median} \PY{o}{=} \PY{l+s}{\PYZdq{}}\PY{l+s}{blue\PYZdq{}}\PY{p}{)}\PY{p}{)}
\end{Verbatim}


    
    
    \begin{center}
    \adjustimage{max size={0.9\linewidth}{0.9\paperheight}}{output_13_1.png}
    \end{center}
    { \hspace*{\fill} \\}
    
    \begin{Verbatim}[commandchars=\\\{\}]
{\color{incolor}In [{\color{incolor}15}]:} \PY{c+c1}{\PYZsh{} Pintamos un histograma por cada variable. Dibujaremos la media y la mediana también.}
         \PY{n+nf}{ggplot}\PY{p}{(}\PY{n}{hayesroth}\PY{p}{,} \PY{n+nf}{aes}\PY{p}{(}\PY{n}{x}\PY{o}{=}\PY{n}{Class}\PY{p}{)}\PY{p}{)}\PY{o}{+}
             \PY{n+nf}{geom\PYZus{}histogram}\PY{p}{(}\PY{n}{bins} \PY{o}{=} \PY{n+nf}{length}\PY{p}{(}\PY{n+nf}{unique}\PY{p}{(}\PY{n}{hayesroth}\PY{o}{\PYZdl{}}\PY{n}{Class}\PY{p}{)}\PY{p}{)}\PY{p}{,} \PY{n}{fill}\PY{o}{=}\PY{l+s}{\PYZdq{}}\PY{l+s}{lightblue\PYZdq{}}\PY{p}{,} \PY{n}{color}\PY{o}{=}\PY{l+s}{\PYZdq{}}\PY{l+s}{black\PYZdq{}}\PY{p}{)}\PY{o}{+}
             \PY{n+nf}{geom\PYZus{}vline}\PY{p}{(}\PY{n+nf}{aes}\PY{p}{(}\PY{n}{xintercept} \PY{o}{=} \PY{n+nf}{mean}\PY{p}{(}\PY{n}{Class}\PY{p}{)}\PY{p}{,}\PY{n}{color} \PY{o}{=} \PY{l+s}{\PYZdq{}}\PY{l+s}{mean\PYZdq{}}\PY{p}{)}\PY{p}{,} 
                      \PY{n}{linetype} \PY{o}{=} \PY{l+s}{\PYZdq{}}\PY{l+s}{dashed\PYZdq{}}\PY{p}{,} \PY{n}{size} \PY{o}{=} \PY{l+m}{0.6}\PY{p}{)} \PY{o}{+}
             \PY{n+nf}{geom\PYZus{}vline}\PY{p}{(}\PY{n+nf}{aes}\PY{p}{(}\PY{n}{xintercept} \PY{o}{=} \PY{n+nf}{median}\PY{p}{(}\PY{n}{Class}\PY{p}{)}\PY{p}{,}\PY{n}{color}\PY{o}{=}\PY{l+s}{\PYZdq{}}\PY{l+s}{median\PYZdq{}}\PY{p}{)}\PY{p}{,}
                           \PY{n}{linetype} \PY{o}{=} \PY{l+s}{\PYZdq{}}\PY{l+s}{dashed\PYZdq{}}\PY{p}{,} \PY{n}{size} \PY{o}{=} \PY{l+m}{0.6} \PY{p}{)} \PY{o}{+}
             \PY{n+nf}{scale\PYZus{}color\PYZus{}manual}\PY{p}{(}\PY{n}{name} \PY{o}{=} \PY{l+s}{\PYZdq{}}\PY{l+s}{statistics\PYZdq{}}\PY{p}{,} \PY{n}{values} \PY{o}{=} \PY{n+nf}{c}\PY{p}{(}\PY{n}{mean} \PY{o}{=} \PY{l+s}{\PYZdq{}}\PY{l+s}{\PYZsh{}FC4E07\PYZdq{}}\PY{p}{,} \PY{n}{median} \PY{o}{=} \PY{l+s}{\PYZdq{}}\PY{l+s}{blue\PYZdq{}}\PY{p}{)}\PY{p}{)}
\end{Verbatim}


    
    
    \begin{center}
    \adjustimage{max size={0.9\linewidth}{0.9\paperheight}}{output_14_1.png}
    \end{center}
    { \hspace*{\fill} \\}
    
    Como se puede ver en las gráficas, nos encontramos ante un problema con
desvalanceo de clases, ya que la variable \emph{Class} cuenta con un
número menor de ejemplos de la clase 3 que para las otras clases.
También podemos ver que en las variables \emph{Age},
\emph{EducationalLevel} y \emph{MaritalStatus} existe un número mayor de
ejemplos sobre los niveles 1 y 2 que para los niveles 3 y 4; para la
variable \emph{Hobby} pasa lo mismo pero del nivel 1 sobre los niveles 2
y 3.

Lo siguiente que haremos será dibujar cada una de las variables
predictoras según la clase, de esta forma podremos ver si los datos de
las diferentes clases están bien diferenciados unos de los otros.

    \begin{Verbatim}[commandchars=\\\{\}]
{\color{incolor}In [{\color{incolor}16}]:} \PY{c+c1}{\PYZsh{} Dibujamos los datos y pintamos dependiendo de la clase a la que pertenezcan.}
         \PY{n+nf}{plot}\PY{p}{(}\PY{n}{hayesroth}\PY{n}{[}\PY{p}{,}\PY{l+m}{1}\PY{o}{:}\PY{l+m}{4}\PY{n}{]}\PY{p}{,}\PY{n}{col}\PY{o}{=}\PY{n}{hayesroth}\PY{n}{[}\PY{p}{,}\PY{l+m}{5}\PY{n}{]}\PY{p}{)}
\end{Verbatim}


    \begin{center}
    \adjustimage{max size={0.9\linewidth}{0.9\paperheight}}{output_16_0.png}
    \end{center}
    { \hspace*{\fill} \\}
    
    \begin{Verbatim}[commandchars=\\\{\}]
{\color{incolor}In [{\color{incolor}17}]:} \PY{n+nf}{ggplot}\PY{p}{(}\PY{n}{hayesroth}\PY{p}{,}\PY{n+nf}{aes}\PY{p}{(}\PY{n}{y}\PY{o}{=}\PY{n}{Hobby}\PY{p}{,}\PY{n}{x}\PY{o}{=}\PY{n}{Age}\PY{p}{)}\PY{p}{)}\PY{o}{+}\PY{n+nf}{geom\PYZus{}point}\PY{p}{(}\PY{n+nf}{aes}\PY{p}{(}\PY{n}{col}\PY{o}{=}\PY{n+nf}{as.factor}\PY{p}{(}\PY{n}{Class}\PY{p}{)}\PY{p}{)}\PY{p}{,}\PY{n}{size}\PY{o}{=}\PY{l+m}{2}\PY{p}{)}\PY{o}{+} \PY{n+nf}{facet\PYZus{}grid}\PY{p}{(}\PY{o}{\PYZti{}}\PY{n}{Class}\PY{p}{)}
\end{Verbatim}


    
    
    \begin{center}
    \adjustimage{max size={0.9\linewidth}{0.9\paperheight}}{output_17_1.png}
    \end{center}
    { \hspace*{\fill} \\}
    
    \begin{Verbatim}[commandchars=\\\{\}]
{\color{incolor}In [{\color{incolor}18}]:} \PY{n+nf}{ggplot}\PY{p}{(}\PY{n}{hayesroth}\PY{p}{,}\PY{n+nf}{aes}\PY{p}{(}\PY{n}{y}\PY{o}{=}\PY{n}{Hobby}\PY{p}{,}\PY{n}{x}\PY{o}{=}\PY{n}{EducationalLevel}\PY{p}{)}\PY{p}{)}\PY{o}{+}\PY{n+nf}{geom\PYZus{}point}\PY{p}{(}\PY{n+nf}{aes}\PY{p}{(}\PY{n}{col}\PY{o}{=}\PY{n+nf}{as.factor}\PY{p}{(}\PY{n}{Class}\PY{p}{)}\PY{p}{)}\PY{p}{,}\PY{n}{size}\PY{o}{=}\PY{l+m}{2}\PY{p}{)}\PY{o}{+} \PY{n+nf}{facet\PYZus{}grid}\PY{p}{(}\PY{o}{\PYZti{}}\PY{n}{Class}\PY{p}{)}
\end{Verbatim}


    
    
    \begin{center}
    \adjustimage{max size={0.9\linewidth}{0.9\paperheight}}{output_18_1.png}
    \end{center}
    { \hspace*{\fill} \\}
    
    \begin{Verbatim}[commandchars=\\\{\}]
{\color{incolor}In [{\color{incolor}19}]:} \PY{n+nf}{ggplot}\PY{p}{(}\PY{n}{hayesroth}\PY{p}{,}\PY{n+nf}{aes}\PY{p}{(}\PY{n}{y}\PY{o}{=}\PY{n}{Hobby}\PY{p}{,}\PY{n}{x}\PY{o}{=}\PY{n}{MaritalStatus}\PY{p}{)}\PY{p}{)}\PY{o}{+}\PY{n+nf}{geom\PYZus{}point}\PY{p}{(}\PY{n+nf}{aes}\PY{p}{(}\PY{n}{col}\PY{o}{=}\PY{n+nf}{as.factor}\PY{p}{(}\PY{n}{Class}\PY{p}{)}\PY{p}{)}\PY{p}{,}\PY{n}{size}\PY{o}{=}\PY{l+m}{2}\PY{p}{)} \PY{o}{+} \PY{n+nf}{facet\PYZus{}grid}\PY{p}{(}\PY{o}{\PYZti{}}\PY{n}{Class}\PY{p}{)}
\end{Verbatim}


    
    
    \begin{center}
    \adjustimage{max size={0.9\linewidth}{0.9\paperheight}}{output_19_1.png}
    \end{center}
    { \hspace*{\fill} \\}
    
    \begin{Verbatim}[commandchars=\\\{\}]
{\color{incolor}In [{\color{incolor}20}]:} \PY{n+nf}{ggplot}\PY{p}{(}\PY{n}{hayesroth}\PY{p}{,}\PY{n+nf}{aes}\PY{p}{(}\PY{n}{y}\PY{o}{=}\PY{n}{Age}\PY{p}{,}\PY{n}{x}\PY{o}{=}\PY{n}{EducationalLevel}\PY{p}{)}\PY{p}{)}\PY{o}{+}\PY{n+nf}{geom\PYZus{}point}\PY{p}{(}\PY{n+nf}{aes}\PY{p}{(}\PY{n}{col}\PY{o}{=}\PY{n+nf}{as.factor}\PY{p}{(}\PY{n}{Class}\PY{p}{)}\PY{p}{)}\PY{p}{,}\PY{n}{size}\PY{o}{=}\PY{l+m}{2}\PY{p}{)} \PY{o}{+} \PY{n+nf}{facet\PYZus{}grid}\PY{p}{(}\PY{o}{\PYZti{}}\PY{n}{Class}\PY{p}{)}
\end{Verbatim}


    
    
    \begin{center}
    \adjustimage{max size={0.9\linewidth}{0.9\paperheight}}{output_20_1.png}
    \end{center}
    { \hspace*{\fill} \\}
    
    \begin{Verbatim}[commandchars=\\\{\}]
{\color{incolor}In [{\color{incolor}21}]:} \PY{n+nf}{ggplot}\PY{p}{(}\PY{n}{hayesroth}\PY{p}{,}\PY{n+nf}{aes}\PY{p}{(}\PY{n}{y}\PY{o}{=}\PY{n}{Age}\PY{p}{,}\PY{n}{x}\PY{o}{=}\PY{n}{MaritalStatus}\PY{p}{)}\PY{p}{)}\PY{o}{+}\PY{n+nf}{geom\PYZus{}point}\PY{p}{(}\PY{n+nf}{aes}\PY{p}{(}\PY{n}{col}\PY{o}{=}\PY{n+nf}{as.factor}\PY{p}{(}\PY{n}{Class}\PY{p}{)}\PY{p}{)}\PY{p}{,}\PY{n}{size}\PY{o}{=}\PY{l+m}{2}\PY{p}{)}\PY{o}{+} \PY{n+nf}{facet\PYZus{}grid}\PY{p}{(}\PY{o}{\PYZti{}}\PY{n}{Class}\PY{p}{)}
\end{Verbatim}


    
    
    \begin{center}
    \adjustimage{max size={0.9\linewidth}{0.9\paperheight}}{output_21_1.png}
    \end{center}
    { \hspace*{\fill} \\}
    
    \begin{Verbatim}[commandchars=\\\{\}]
{\color{incolor}In [{\color{incolor}22}]:} \PY{n+nf}{ggplot}\PY{p}{(}\PY{n}{hayesroth}\PY{p}{,}\PY{n+nf}{aes}\PY{p}{(}\PY{n}{y}\PY{o}{=}\PY{n}{MaritalStatus}\PY{p}{,}\PY{n}{x}\PY{o}{=}\PY{n}{EducationalLevel}\PY{p}{)}\PY{p}{)}\PY{o}{+}
             \PY{n+nf}{geom\PYZus{}point}\PY{p}{(}\PY{n+nf}{aes}\PY{p}{(}\PY{n}{col}\PY{o}{=}\PY{n+nf}{as.factor}\PY{p}{(}\PY{n}{Class}\PY{p}{)}\PY{p}{)}\PY{p}{,}\PY{n}{size}\PY{o}{=}\PY{l+m}{2}\PY{p}{)}\PY{o}{+} \PY{n+nf}{facet\PYZus{}grid}\PY{p}{(}\PY{o}{\PYZti{}}\PY{n}{Class}\PY{p}{)}
\end{Verbatim}


    
    
    \begin{center}
    \adjustimage{max size={0.9\linewidth}{0.9\paperheight}}{output_22_1.png}
    \end{center}
    { \hspace*{\fill} \\}
    
    \begin{Verbatim}[commandchars=\\\{\}]
{\color{incolor}In [{\color{incolor}23}]:} \PY{n+nf}{library}\PY{p}{(}\PY{n}{reshape2}\PY{p}{)}
         \PY{n}{temp} \PY{o}{=} \PY{n}{hayesroth}
         \PY{n}{nhayes} \PY{o}{\PYZlt{}\PYZhy{}} \PY{n+nf}{melt}\PY{p}{(}\PY{n}{temp}\PY{p}{,} \PY{n}{id.vars} \PY{o}{=} \PY{l+s}{\PYZdq{}}\PY{l+s}{Class\PYZdq{}}\PY{p}{)}
\end{Verbatim}


    \begin{Verbatim}[commandchars=\\\{\}]
{\color{incolor}In [{\color{incolor}24}]:} \PY{n+nf}{ggplot}\PY{p}{(}\PY{n}{nhayes}\PY{p}{,} \PY{n+nf}{aes}\PY{p}{(}\PY{n}{x}\PY{o}{=}\PY{n}{variable}\PY{p}{,} \PY{n}{y}\PY{o}{=}\PY{n}{value}\PY{p}{,} \PY{n}{group}\PY{o}{=}\PY{n}{Class}\PY{p}{)}\PY{p}{)} \PY{o}{+} 
             \PY{n+nf}{geom\PYZus{}point}\PY{p}{(}\PY{n+nf}{aes}\PY{p}{(}\PY{n}{color}\PY{o}{=}\PY{n+nf}{as.factor}\PY{p}{(}\PY{n}{Class}\PY{p}{)}\PY{p}{)}\PY{p}{)} \PY{o}{+}
                         \PY{n+nf}{facet\PYZus{}grid}\PY{p}{(}\PY{o}{\PYZti{}} \PY{n}{Class}\PY{p}{)} \PY{o}{+}
             \PY{n+nf}{theme}\PY{p}{(}\PY{n}{axis.text.x} \PY{o}{=} \PY{n+nf}{element\PYZus{}text}\PY{p}{(}\PY{n}{angle} \PY{o}{=} \PY{l+m}{45}\PY{p}{,} \PY{n}{hjust} \PY{o}{=} \PY{l+m}{1}\PY{p}{)}\PY{p}{)}
\end{Verbatim}


    
    
    \begin{center}
    \adjustimage{max size={0.9\linewidth}{0.9\paperheight}}{output_24_1.png}
    \end{center}
    { \hspace*{\fill} \\}
    
    Como se puede ver en las gráficas anteriores, existen valores de todos
los tipos en todas las clases, esto nos dice que la separación entre
dichas clases es bastante mala, y por ello será bastante difícil obtener
un buen clasificador.La diferencia más clara que se puede ver es que los
elementos que toman valores 4 para las variables \emph{Age},
\emph{EducationalLevel} y \emph{MaritalStatus} pertenecen siempre a la
tercera clase.

También podemos ver como la variable \emph{Hobby} presenta valores en
todas las clases para todos sus valores, por lo que no merece la pena
utilizarla en la predicción (como está descrito en el apartado de
información, la variable \emph{Hobby} es una variable generada de forma
aleatoria para introducir ruido en los datos). Para el resto de los
datos, podemos ver que existen algunas diferencias entre los datos de
las diferentes clases, por ejemplo, ningún dato con \emph{MaritalStatus}
y \emph{EducationalLevel} 1 pertenecen a la segunda clase o los datos de
\emph{EducationalLevel} 2 ó 3 y \emph{Age} 2 nunca pertencen a la
primera clase. Con las pequeñas diferencias que hay entre las variables
\emph{Age}, \emph{EducationalLevel} y \emph{MaritalStatus} nuestro
clasificador deberá de conseguir separar las 3 clases que existen.

    Probaremos también ha transformar los datos con PCA, para ver si
conseguimos una mejor separación de estos.

    \begin{Verbatim}[commandchars=\\\{\}]
{\color{incolor}In [{\color{incolor}25}]:} \PY{n+nf}{require}\PY{p}{(}\PY{n}{caret}\PY{p}{)}
         \PY{n}{temp} \PY{o}{=} \PY{n}{hayesroth}
         \PY{n}{temp} \PY{o}{=} \PY{n}{temp}\PY{n}{[}\PY{p}{,}\PY{l+m}{\PYZhy{}1}\PY{n}{]}
         \PY{n}{transform} \PY{o}{=} \PY{n+nf}{preProcess}\PY{p}{(}\PY{n}{temp}\PY{n}{[1}\PY{o}{:}\PY{l+m}{3}\PY{n}{]}\PY{p}{,}\PY{n}{method}\PY{o}{=}\PY{n+nf}{c}\PY{p}{(}\PY{l+s}{\PYZdq{}}\PY{l+s}{BoxCox\PYZdq{}}\PY{p}{,} \PY{l+s}{\PYZdq{}}\PY{l+s}{center\PYZdq{}}\PY{p}{,} 
                                     \PY{l+s}{\PYZdq{}}\PY{l+s}{scale\PYZdq{}}\PY{p}{,} \PY{l+s}{\PYZdq{}}\PY{l+s}{pca\PYZdq{}}\PY{p}{)}\PY{p}{)}
         \PY{n}{PCA} \PY{o}{=} \PY{n+nf}{predict}\PY{p}{(}\PY{n}{transform}\PY{p}{,} \PY{n}{temp}\PY{n}{[1}\PY{o}{:}\PY{l+m}{3}\PY{n}{]}\PY{p}{)}
         \PY{n+nf}{head}\PY{p}{(}\PY{n}{PCA}\PY{p}{)}
\end{Verbatim}


    \begin{Verbatim}[commandchars=\\\{\}]
Loading required package: caret
Loading required package: lattice

    \end{Verbatim}

    \begin{tabular}{r|lll}
 PC1 & PC2 & PC3\\
\hline
	  1.018395e+00 & -0.5879708    & -1.2103330   \\
	 -5.957231e-01 & -1.5198825    &  0.1075892   \\
	 -2.036791e+00 & -1.1759416    & -0.3788167   \\
	 -8.326673e-17 &  1.1759416    &  1.2842159   \\
	  4.226723e-01 & -2.1078533    &  0.9391055   \\
	 -5.957231e-01 & -1.5198825    &  0.1075892   \\
\end{tabular}


    
    \begin{Verbatim}[commandchars=\\\{\}]
{\color{incolor}In [{\color{incolor}26}]:} \PY{n+nf}{pairs}\PY{p}{(}\PY{n}{PCA}\PY{p}{,} \PY{n}{col}\PY{o}{=}\PY{n}{temp}\PY{o}{\PYZdl{}}\PY{n}{Class}\PY{p}{)}
\end{Verbatim}


    \begin{center}
    \adjustimage{max size={0.9\linewidth}{0.9\paperheight}}{output_28_0.png}
    \end{center}
    { \hspace*{\fill} \\}
    
    Como se puede ver, PCA no consigue separar los datos ni reduce en número
de variables del problema, por lo que no utilizaremos esta
transformación. Lo siguiente que vamos a hacer es comprobar si existe
alguna correlación entre las variables.

    \begin{Verbatim}[commandchars=\\\{\}]
{\color{incolor}In [{\color{incolor}105}]:} \PY{n+nf}{pairs}\PY{p}{(}\PY{n}{hayesroth}\PY{p}{,}\PY{n}{col}\PY{o}{=}\PY{n}{hayesroth}\PY{n}{[}\PY{p}{,}\PY{l+m}{5}\PY{n}{]}\PY{p}{)}
\end{Verbatim}


    \begin{center}
    \adjustimage{max size={0.9\linewidth}{0.9\paperheight}}{output_30_0.png}
    \end{center}
    { \hspace*{\fill} \\}
    
    \begin{Verbatim}[commandchars=\\\{\}]
{\color{incolor}In [{\color{incolor}106}]:} \PY{n+nf}{cor}\PY{p}{(}\PY{n}{hayesroth}\PY{p}{)}
\end{Verbatim}


    \begin{tabular}{r|lllll}
  & Hobby & Age & EducationalLevel & MaritalStatus & Class\\
\hline
	Hobby &  1.00000000 & 0.04251185  & -0.05374139 & -0.05374139 & 0.0709883  \\
	Age &  0.04251185 & 1.00000000  &  0.01358113 &  0.01358113 & 0.4141117  \\
	EducationalLevel & -0.05374139 & 0.01358113  &  1.00000000 &  0.01358113 & 0.3961654  \\
	MaritalStatus & -0.05374139 & 0.01358113  &  0.01358113 &  1.00000000 & 0.3782190  \\
	Class &  0.07098830 & 0.41411171  &  0.39616537 &  0.37821903 & 1.0000000  \\
\end{tabular}


    
    Como podemos ver, no existen correlaciones entre las variables
predictoras ni con la variable predictora, por ello no podemos crear
variables nuevas a partir de estas ni eliminar ninguna.

    \hypertarget{creaciuxf3n-de-modelos-de-predicciuxf3n}{%
\subsection{Creación de modelos de
predicción}\label{creaciuxf3n-de-modelos-de-predicciuxf3n}}

En este apartado se crearán tres modelos diferentes para predecir la
clase a la que pertenecen los datos y se realizará validación cruzada
con el mejor modelo encontrado para cada algoritmo que utilizemos. Los
algoritmos que se van a probar son \emph{KNN}, \emph{LDA} y \emph{QDA}.
Las variables que utilizaremmos con estos algoritmos serán \emph{Age},
\emph{MaritalStatus} y \emph{EducationalLevel}.

    \hypertarget{creaciuxf3n-de-modelos-con-el-algoritmo-knn}{%
\subsubsection{Creación de modelos con el algoritmo
KNN}\label{creaciuxf3n-de-modelos-con-el-algoritmo-knn}}

En este apartado utilizaremos el algoritmo KNN para obtener un modelo,
también probaremos con diferentes tamaños de k para ver cuales de estos
se ajustan mejor al problema. Para ello, utilizaremos la librería
\emph{caret} que no permite especificar los diferentes tamaños de k que
nos interesen.

    \begin{Verbatim}[commandchars=\\\{\}]
{\color{incolor}In [{\color{incolor}27}]:} \PY{c+c1}{\PYZsh{} cargamos la librería.}
         \PY{n+nf}{require}\PY{p}{(}\PY{n}{caret}\PY{p}{)}
\end{Verbatim}


    \begin{Verbatim}[commandchars=\\\{\}]
{\color{incolor}In [{\color{incolor}28}]:} \PY{c+c1}{\PYZsh{} Para k utilizaremos los valores impares del 3 al 15, para evitar empates dentro de knn.}
         \PY{n}{ks} \PY{o}{=} \PY{l+m}{3}\PY{o}{:}\PY{l+m}{15}
         \PY{n}{ks} \PY{o}{=} \PY{n}{ks}\PY{n}{[ks}\PY{o}{\PYZpc{}\PYZpc{}}\PY{l+m}{2} \PY{o}{!=} \PY{l+m}{0}\PY{n}{]}
         
         \PY{c+c1}{\PYZsh{} Separamos las variables de entrada y las clases en dos variables distintas para poder utilizarlo con caret.}
         \PY{n}{hr.train} \PY{o}{=} \PY{n}{hayesroth}\PY{n}{[}\PY{p}{,}\PY{n+nf}{c}\PY{p}{(}\PY{l+s}{\PYZdq{}}\PY{l+s}{Age\PYZdq{}}\PY{p}{,}\PY{l+s}{\PYZdq{}}\PY{l+s}{EducationalLevel\PYZdq{}}\PY{p}{,}\PY{l+s}{\PYZdq{}}\PY{l+s}{MaritalStatus\PYZdq{}}\PY{p}{)}\PY{n}{]}
         \PY{n}{hr.labels.train} \PY{o}{=} \PY{n}{hayesroth}\PY{n}{[}\PY{p}{,}\PY{l+s}{\PYZdq{}}\PY{l+s}{Class\PYZdq{}}\PY{n}{]}
         \PY{n}{hr.labels.train} \PY{o}{=} \PY{n+nf}{factor}\PY{p}{(}\PY{n}{hr.labels.train}\PY{p}{,} \PY{n}{levels}\PY{o}{=}\PY{n+nf}{c}\PY{p}{(}\PY{l+m}{1}\PY{p}{,}\PY{l+m}{2}\PY{p}{,}\PY{l+m}{3}\PY{p}{)}\PY{p}{)}
\end{Verbatim}


    \begin{Verbatim}[commandchars=\\\{\}]
{\color{incolor}In [{\color{incolor}29}]:} \PY{c+c1}{\PYZsh{} Normalizamos los datos ya que estamos trabajando con knn, aunque al tener}
         \PY{c+c1}{\PYZsh{} la misma escala en todas las variables de entrada no es necesario.}
         \PY{n}{hr.train} \PY{o}{=} \PY{n+nf}{as.data.frame}\PY{p}{(}\PY{n+nf}{lapply}\PY{p}{(}\PY{n}{hr.train}\PY{p}{,}
                                        \PY{n}{scale}\PY{p}{,} \PY{n}{center} \PY{o}{=} \PY{k+kc}{TRUE}\PY{p}{,} \PY{n}{scale} \PY{o}{=} \PY{k+kc}{TRUE}\PY{p}{)}\PY{p}{)}
\end{Verbatim}


    \begin{Verbatim}[commandchars=\\\{\}]
{\color{incolor}In [{\color{incolor}30}]:} \PY{c+c1}{\PYZsh{} creamos nuestro modelo y obtenemos nuestro mejor k.}
         \PY{n}{knnModel} \PY{o}{\PYZlt{}\PYZhy{}} \PY{n+nf}{train}\PY{p}{(}\PY{n}{hr.train}\PY{p}{,}\PY{n}{hr.labels.train}\PY{p}{,}
                           \PY{n}{method}\PY{o}{=}\PY{l+s}{\PYZdq{}}\PY{l+s}{knn\PYZdq{}}\PY{p}{,} \PY{n}{metric}\PY{o}{=}\PY{l+s}{\PYZdq{}}\PY{l+s}{Accuracy\PYZdq{}}\PY{p}{,}
                           \PY{n}{tuneGrid} \PY{o}{=} \PY{n+nf}{data.frame}\PY{p}{(}\PY{n}{.k}\PY{o}{=}\PY{n}{ks}\PY{p}{)}\PY{p}{)}
         \PY{n}{knnModel}
\end{Verbatim}


    
    \begin{verbatim}
k-Nearest Neighbors 

160 samples
  3 predictor
  3 classes: '1', '2', '3' 

No pre-processing
Resampling: Bootstrapped (25 reps) 
Summary of sample sizes: 160, 160, 160, 160, 160, 160, ... 
Resampling results across tuning parameters:

  k   Accuracy   Kappa      
   3  0.7592192   0.61538285
   5  0.7327564   0.57240739
   7  0.7008101   0.51953449
   9  0.6068701   0.36899478
  11  0.5227671   0.23058421
  13  0.4166538   0.05572926
  15  0.3508964  -0.05400105

Accuracy was used to select the optimal model using the largest value.
The final value used for the model was k = 3.
    \end{verbatim}

    
    \begin{Verbatim}[commandchars=\\\{\}]
{\color{incolor}In [{\color{incolor}31}]:} \PY{c+c1}{\PYZsh{} dibujamos los resultados de cada k.}
         \PY{n}{resultados\PYZus{}k} \PY{o}{=} \PY{n}{knnModel}\PY{o}{\PYZdl{}}\PY{n}{results}\PY{n}{[}\PY{p}{,}\PY{n+nf}{c}\PY{p}{(}\PY{l+s}{\PYZdq{}}\PY{l+s}{k\PYZdq{}}\PY{p}{,}\PY{l+s}{\PYZdq{}}\PY{l+s}{Accuracy\PYZdq{}}\PY{p}{)}\PY{n}{]}
\end{Verbatim}


    \begin{Verbatim}[commandchars=\\\{\}]
{\color{incolor}In [{\color{incolor}32}]:} \PY{c+c1}{\PYZsh{} cargamos la librería ggplot}
         \PY{n+nf}{require}\PY{p}{(}\PY{n}{ggplot2}\PY{p}{)}
\end{Verbatim}


    \begin{Verbatim}[commandchars=\\\{\}]
{\color{incolor}In [{\color{incolor}33}]:} \PY{n+nf}{ggplot}\PY{p}{(}\PY{n}{resultados\PYZus{}k}\PY{p}{,} \PY{n+nf}{aes}\PY{p}{(}\PY{n}{x}\PY{o}{=}\PY{n}{k}\PY{p}{,} \PY{n}{y}\PY{o}{=}\PY{n}{Accuracy}\PY{p}{)}\PY{p}{)} \PY{o}{+} \PY{n+nf}{geom\PYZus{}point}\PY{p}{(}\PY{n}{col}\PY{o}{=}\PY{l+s}{\PYZdq{}}\PY{l+s}{deepskyblue\PYZdq{}}\PY{p}{)} \PY{o}{+}
             \PY{n+nf}{geom\PYZus{}line}\PY{p}{(}\PY{p}{)} \PY{o}{+} \PY{n+nf}{scale\PYZus{}x\PYZus{}continuous}\PY{p}{(}\PY{n}{breaks}\PY{o}{=}\PY{n}{resultados\PYZus{}k}\PY{o}{\PYZdl{}}\PY{n}{k}\PY{p}{)}
\end{Verbatim}


    
    
    \begin{center}
    \adjustimage{max size={0.9\linewidth}{0.9\paperheight}}{output_41_1.png}
    \end{center}
    { \hspace*{\fill} \\}
    
    Como se puede ver en la gráfica, el mejor resultado se obtiene para
\emph{k} igual a 3, nos quedaremos con este modelo para hacer validación
cruzada (ya que caret crea el modelo final con el k que mejores valores
a obtenido).

    \begin{Verbatim}[commandchars=\\\{\}]
{\color{incolor}In [{\color{incolor}34}]:} \PY{n}{knnModel} \PY{o}{\PYZlt{}\PYZhy{}} \PY{n+nf}{train}\PY{p}{(}\PY{n}{hr.train}\PY{p}{,}\PY{n}{hr.labels.train}\PY{p}{,}
                           \PY{n}{method}\PY{o}{=}\PY{l+s}{\PYZdq{}}\PY{l+s}{knn\PYZdq{}}\PY{p}{,} \PY{n}{metric}\PY{o}{=}\PY{l+s}{\PYZdq{}}\PY{l+s}{Accuracy\PYZdq{}}\PY{p}{,}
                           \PY{n}{tuneGrid} \PY{o}{=} \PY{n+nf}{data.frame}\PY{p}{(}\PY{n}{.k}\PY{o}{=}\PY{l+m}{3}\PY{p}{)}\PY{p}{)}
         \PY{n}{knnModel}
         
         \PY{n}{knnPred} \PY{o}{=} \PY{n+nf}{predict}\PY{p}{(}\PY{n}{knnModel}\PY{p}{,}\PY{n}{newdata}\PY{o}{=}\PY{n}{hr.train}\PY{p}{)} 
         \PY{n}{acc}\PY{o}{=}\PY{n+nf}{postResample}\PY{p}{(}\PY{n}{pred} \PY{o}{=} \PY{n}{knnPred}\PY{p}{,} \PY{n}{obs} \PY{o}{=} \PY{n}{hr.labels.train}\PY{p}{)}\PY{n}{[1}\PY{n}{]}
\end{Verbatim}


    
    \begin{verbatim}
k-Nearest Neighbors 

160 samples
  3 predictor
  3 classes: '1', '2', '3' 

No pre-processing
Resampling: Bootstrapped (25 reps) 
Summary of sample sizes: 160, 160, 160, 160, 160, 160, ... 
Resampling results:

  Accuracy  Kappa    
  0.756918  0.6094881

Tuning parameter 'k' was held constant at a value of 3
    \end{verbatim}

    
    \begin{Verbatim}[commandchars=\\\{\}]
{\color{incolor}In [{\color{incolor}101}]:} \PY{n}{nombre} \PY{o}{\PYZlt{}\PYZhy{}} \PY{l+s}{\PYZdq{}}\PY{l+s}{hayes\PYZhy{}roth/hayes\PYZhy{}roth\PYZdq{}}
          
          \PY{n}{run\PYZus{}knn\PYZus{}fold} \PY{o}{\PYZlt{}\PYZhy{}} \PY{n+nf}{function}\PY{p}{(}\PY{n}{i}\PY{p}{,} \PY{n}{x}\PY{p}{,} \PY{n}{tt} \PY{o}{=} \PY{l+s}{\PYZdq{}}\PY{l+s}{test\PYZdq{}}\PY{p}{)} \PY{p}{\PYZob{}}
              \PY{n}{file} \PY{o}{\PYZlt{}\PYZhy{}} \PY{n+nf}{paste}\PY{p}{(}\PY{n}{x}\PY{p}{,} \PY{l+s}{\PYZdq{}}\PY{l+s}{\PYZhy{}10\PYZhy{}\PYZdq{}}\PY{p}{,} \PY{n}{i}\PY{p}{,} \PY{l+s}{\PYZdq{}}\PY{l+s}{tra.dat\PYZdq{}}\PY{p}{,} \PY{n}{sep}\PY{o}{=}\PY{l+s}{\PYZdq{}}\PY{l+s}{\PYZdq{}}\PY{p}{)}
              \PY{n}{x\PYZus{}tra} \PY{o}{\PYZlt{}\PYZhy{}} \PY{n+nf}{read.csv}\PY{p}{(}\PY{n}{file}\PY{p}{,} \PY{n}{comment.char}\PY{o}{=}\PY{l+s}{\PYZdq{}}\PY{l+s}{@\PYZdq{}}\PY{p}{,} \PY{n}{header}\PY{o}{=}\PY{k+kc}{FALSE}\PY{p}{)}
              \PY{n}{file} \PY{o}{\PYZlt{}\PYZhy{}} \PY{n+nf}{paste}\PY{p}{(}\PY{n}{x}\PY{p}{,} \PY{l+s}{\PYZdq{}}\PY{l+s}{\PYZhy{}10\PYZhy{}\PYZdq{}}\PY{p}{,} \PY{n}{i}\PY{p}{,} \PY{l+s}{\PYZdq{}}\PY{l+s}{tst.dat\PYZdq{}}\PY{p}{,} \PY{n}{sep}\PY{o}{=}\PY{l+s}{\PYZdq{}}\PY{l+s}{\PYZdq{}}\PY{p}{)}
              \PY{n}{x\PYZus{}tst} \PY{o}{\PYZlt{}\PYZhy{}} \PY{n+nf}{read.csv}\PY{p}{(}\PY{n}{file}\PY{p}{,} \PY{n}{comment.char}\PY{o}{=}\PY{l+s}{\PYZdq{}}\PY{l+s}{@\PYZdq{}}\PY{p}{,} \PY{n}{header}\PY{o}{=}\PY{k+kc}{FALSE}\PY{p}{)}
              \PY{n}{In} \PY{o}{\PYZlt{}\PYZhy{}} \PY{n+nf}{length}\PY{p}{(}\PY{n+nf}{names}\PY{p}{(}\PY{n}{x\PYZus{}tra}\PY{p}{)}\PY{p}{)} \PY{o}{\PYZhy{}} \PY{l+m}{1}
              \PY{n+nf}{names}\PY{p}{(}\PY{n}{x\PYZus{}tra}\PY{p}{)}\PY{n}{[1}\PY{o}{:}\PY{n}{In}\PY{n}{]} \PY{o}{\PYZlt{}\PYZhy{}} \PY{n+nf}{paste }\PY{p}{(}\PY{l+s}{\PYZdq{}}\PY{l+s}{X\PYZdq{}}\PY{p}{,} \PY{l+m}{1}\PY{o}{:}\PY{n}{In}\PY{p}{,} \PY{n}{sep}\PY{o}{=}\PY{l+s}{\PYZdq{}}\PY{l+s}{\PYZdq{}}\PY{p}{)}
              \PY{n+nf}{names}\PY{p}{(}\PY{n}{x\PYZus{}tra}\PY{p}{)}\PY{n}{[In}\PY{l+m}{+1}\PY{n}{]} \PY{o}{\PYZlt{}\PYZhy{}} \PY{l+s}{\PYZdq{}}\PY{l+s}{Y\PYZdq{}}
              \PY{n+nf}{names}\PY{p}{(}\PY{n}{x\PYZus{}tst}\PY{p}{)}\PY{n}{[1}\PY{o}{:}\PY{n}{In}\PY{n}{]} \PY{o}{\PYZlt{}\PYZhy{}} \PY{n+nf}{paste }\PY{p}{(}\PY{l+s}{\PYZdq{}}\PY{l+s}{X\PYZdq{}}\PY{p}{,} \PY{l+m}{1}\PY{o}{:}\PY{n}{In}\PY{p}{,} \PY{n}{sep}\PY{o}{=}\PY{l+s}{\PYZdq{}}\PY{l+s}{\PYZdq{}}\PY{p}{)}
              \PY{n+nf}{names}\PY{p}{(}\PY{n}{x\PYZus{}tst}\PY{p}{)}\PY{n}{[In}\PY{l+m}{+1}\PY{n}{]} \PY{o}{\PYZlt{}\PYZhy{}} \PY{l+s}{\PYZdq{}}\PY{l+s}{Y\PYZdq{}}
              \PY{n+nf}{if }\PY{p}{(}\PY{n}{tt} \PY{o}{==} \PY{l+s}{\PYZdq{}}\PY{l+s}{train\PYZdq{}}\PY{p}{)} \PY{p}{\PYZob{}}
                  \PY{n}{test} \PY{o}{\PYZlt{}\PYZhy{}} \PY{n}{x\PYZus{}tra}
              \PY{p}{\PYZcb{}}
              \PY{n}{else} \PY{p}{\PYZob{}}
                  \PY{n}{test} \PY{o}{\PYZlt{}\PYZhy{}} \PY{n}{x\PYZus{}tst}
              \PY{p}{\PYZcb{}}
              
              \PY{n}{hr.train} \PY{o}{=} \PY{n}{x\PYZus{}tra}\PY{n}{[}\PY{p}{,}\PY{n+nf}{c}\PY{p}{(}\PY{l+s}{\PYZdq{}}\PY{l+s}{X2\PYZdq{}}\PY{p}{,}\PY{l+s}{\PYZdq{}}\PY{l+s}{X3\PYZdq{}}\PY{p}{,}\PY{l+s}{\PYZdq{}}\PY{l+s}{X4\PYZdq{}}\PY{p}{)}\PY{n}{]}
              \PY{n}{hr.labels.train} \PY{o}{=} \PY{n}{x\PYZus{}tra}\PY{n}{[}\PY{p}{,}\PY{l+s}{\PYZdq{}}\PY{l+s}{Y\PYZdq{}}\PY{n}{]}
              \PY{n}{hr.labels.train} \PY{o}{=} \PY{n+nf}{factor}\PY{p}{(}\PY{n}{hr.labels.train}\PY{p}{,} \PY{n}{levels}\PY{o}{=}\PY{n+nf}{c}\PY{p}{(}\PY{l+m}{1}\PY{p}{,}\PY{l+m}{2}\PY{p}{,}\PY{l+m}{3}\PY{p}{)}\PY{p}{)}
              \PY{n}{fitMulti}\PY{o}{=}\PY{n+nf}{train}\PY{p}{(}\PY{n}{hr.train}\PY{p}{,}\PY{n}{hr.labels.train}\PY{p}{,}
                            \PY{n}{method}\PY{o}{=}\PY{l+s}{\PYZdq{}}\PY{l+s}{knn\PYZdq{}}\PY{p}{,} \PY{n}{metric}\PY{o}{=}\PY{l+s}{\PYZdq{}}\PY{l+s}{Accuracy\PYZdq{}}\PY{p}{,}
                            \PY{n}{tuneGrid} \PY{o}{=} \PY{n+nf}{data.frame}\PY{p}{(}\PY{n}{.k}\PY{o}{=}\PY{l+m}{3}\PY{p}{)}\PY{p}{)}
              
              \PY{n}{labels} \PY{o}{=} \PY{n+nf}{factor}\PY{p}{(}\PY{n}{test}\PY{n}{[}\PY{p}{,}\PY{l+s}{\PYZdq{}}\PY{l+s}{Y\PYZdq{}}\PY{n}{]}\PY{p}{,}\PY{n}{levels}\PY{o}{=}\PY{n+nf}{c}\PY{p}{(}\PY{l+m}{1}\PY{p}{,}\PY{l+m}{2}\PY{p}{,}\PY{l+m}{3}\PY{p}{)}\PY{p}{)}
              \PY{n}{yprime}\PY{o}{=}\PY{n+nf}{predict}\PY{p}{(}\PY{n}{fitMulti}\PY{p}{,}\PY{n}{newdata}\PY{o}{=}\PY{n}{test}\PY{n}{[}\PY{p}{,}\PY{n+nf}{c}\PY{p}{(}\PY{l+s}{\PYZdq{}}\PY{l+s}{X2\PYZdq{}}\PY{p}{,}\PY{l+s}{\PYZdq{}}\PY{l+s}{X3\PYZdq{}}\PY{p}{,}\PY{l+s}{\PYZdq{}}\PY{l+s}{X4\PYZdq{}}\PY{p}{)}\PY{n}{]}\PY{p}{)}
              \PY{n}{err} \PY{o}{=} \PY{l+m}{1}\PY{o}{\PYZhy{}}\PY{n+nf}{postResample}\PY{p}{(}\PY{n}{pred} \PY{o}{=} \PY{n}{yprime}\PY{p}{,} \PY{n}{obs} \PY{o}{=} \PY{n}{labels}\PY{p}{)}\PY{n}{[1}\PY{n}{]}
              
          \PY{p}{\PYZcb{}}
          \PY{n}{knnMSEtrain.all}\PY{o}{\PYZlt{}\PYZhy{}}\PY{n+nf}{mean}\PY{p}{(}\PY{n+nf}{sapply}\PY{p}{(}\PY{l+m}{1}\PY{o}{:}\PY{l+m}{10}\PY{p}{,}\PY{n}{run\PYZus{}knn\PYZus{}fold}\PY{p}{,}\PY{n}{nombre}\PY{p}{,}\PY{l+s}{\PYZdq{}}\PY{l+s}{train\PYZdq{}}\PY{p}{)}\PY{p}{)}
          \PY{n}{knnMSEtest.all}\PY{o}{\PYZlt{}\PYZhy{}}\PY{n+nf}{mean}\PY{p}{(}\PY{n+nf}{sapply}\PY{p}{(}\PY{l+m}{1}\PY{o}{:}\PY{l+m}{10}\PY{p}{,}\PY{n}{run\PYZus{}knn\PYZus{}fold}\PY{p}{,}\PY{n}{nombre}\PY{p}{,}\PY{l+s}{\PYZdq{}}\PY{l+s}{test\PYZdq{}}\PY{p}{)}\PY{p}{)}
          \PY{n+nf}{print}\PY{p}{(}\PY{n}{knnMSEtrain.all}\PY{p}{)}
          \PY{n+nf}{print}\PY{p}{(}\PY{n}{knnMSEtest.all}\PY{p}{)}
\end{Verbatim}


    \begin{Verbatim}[commandchars=\\\{\}]
[1] 0.1444444
[1] 0.25625

    \end{Verbatim}

    Como se puede ver por los resultdos obtenidos, el algoritmo de knn tiene
sobreaprendizaje.

    \hypertarget{modelos-con-el-algoritmo-lda}{%
\subsubsection{Modelos con el algoritmo
LDA}\label{modelos-con-el-algoritmo-lda}}

En este apartado utilizaremos el algoritmo LDA para crear un modelo que
prediga la clase a que pertenece cada dato. Antes de crear los modelos
con LDA, comprobaremos si las variables tienen una varianza igual y
tienen una ditribución normal; al igual que pare el algoritmo KNN,
solamente utilizaremos las variables \emph{Age}, \emph{EducationalLevel}
y \emph{MaritalStatus}.

    \begin{Verbatim}[commandchars=\\\{\}]
{\color{incolor}In [{\color{incolor}47}]:} \PY{c+c1}{\PYZsh{} Separamos las variables de entrada y las clases en dos variables distintas para poder utilizarlo con caret.}
         \PY{n}{hr.train} \PY{o}{=} \PY{n}{hayesroth}\PY{n}{[}\PY{p}{,}\PY{n+nf}{c}\PY{p}{(}\PY{l+s}{\PYZdq{}}\PY{l+s}{Age\PYZdq{}}\PY{p}{,}\PY{l+s}{\PYZdq{}}\PY{l+s}{EducationalLevel\PYZdq{}}\PY{p}{,}\PY{l+s}{\PYZdq{}}\PY{l+s}{MaritalStatus\PYZdq{}}\PY{p}{)}\PY{n}{]}
         \PY{n}{hr.labels.train} \PY{o}{=} \PY{n}{hayesroth}\PY{n}{[}\PY{p}{,}\PY{l+s}{\PYZdq{}}\PY{l+s}{Class\PYZdq{}}\PY{n}{]}
         \PY{n}{hr.labels.train} \PY{o}{=} \PY{n+nf}{factor}\PY{p}{(}\PY{n}{hr.labels.train}\PY{p}{,} \PY{n}{levels}\PY{o}{=}\PY{n+nf}{c}\PY{p}{(}\PY{l+m}{1}\PY{p}{,}\PY{l+m}{2}\PY{p}{,}\PY{l+m}{3}\PY{p}{)}\PY{p}{)}
         
         \PY{c+c1}{\PYZsh{} Normalizamos los datos.}
         \PY{n}{hr.train} \PY{o}{=} \PY{n+nf}{as.data.frame}\PY{p}{(}\PY{n+nf}{lapply}\PY{p}{(}\PY{n}{hr.train}\PY{p}{,}
                                        \PY{n}{scale}\PY{p}{,} \PY{n}{center} \PY{o}{=} \PY{k+kc}{TRUE}\PY{p}{,} \PY{n}{scale} \PY{o}{=} \PY{k+kc}{TRUE}\PY{p}{)}\PY{p}{)}
\end{Verbatim}


    \begin{Verbatim}[commandchars=\\\{\}]
{\color{incolor}In [{\color{incolor}48}]:} \PY{c+c1}{\PYZsh{} Cargamos la librería necesaria para comprobar si las distribuciones son normales.}
         \PY{n+nf}{require}\PY{p}{(}\PY{n}{MASS}\PY{p}{)} \PY{c+c1}{\PYZsh{} lda y qda.}
\end{Verbatim}


    \begin{Verbatim}[commandchars=\\\{\}]
{\color{incolor}In [{\color{incolor}49}]:} \PY{c+c1}{\PYZsh{} Comprobamos si tienen una distribución normal y misma varianza.}
         \PY{n+nf}{shapiro.test}\PY{p}{(}\PY{n}{hr.train}\PY{o}{\PYZdl{}}\PY{n}{Age}\PY{p}{)}
         \PY{n+nf}{shapiro.test}\PY{p}{(}\PY{n}{hr.train}\PY{o}{\PYZdl{}}\PY{n}{EducationalLevel}\PY{p}{)}
         \PY{n+nf}{shapiro.test}\PY{p}{(}\PY{n}{hr.train}\PY{o}{\PYZdl{}}\PY{n}{MaritalStatus}\PY{p}{)}
         
         \PY{n+nf}{var}\PY{p}{(}\PY{n}{hr.train}\PY{o}{\PYZdl{}}\PY{n}{Age}\PY{p}{)}
         \PY{n+nf}{var}\PY{p}{(}\PY{n}{hr.train}\PY{o}{\PYZdl{}}\PY{n}{EducationalLevel}\PY{p}{)}
         \PY{n+nf}{var}\PY{p}{(}\PY{n}{hr.train}\PY{o}{\PYZdl{}}\PY{n}{MaritalStatus}\PY{p}{)}
\end{Verbatim}


    
    \begin{verbatim}

	Shapiro-Wilk normality test

data:  hr.train$Age
W = 0.8329, p-value = 3.074e-12

    \end{verbatim}

    
    
    \begin{verbatim}

	Shapiro-Wilk normality test

data:  hr.train$EducationalLevel
W = 0.8329, p-value = 3.074e-12

    \end{verbatim}

    
    
    \begin{verbatim}

	Shapiro-Wilk normality test

data:  hr.train$MaritalStatus
W = 0.8329, p-value = 3.074e-12

    \end{verbatim}

    
    1

    
    1

    
    1

    
    Como se puede ver según los resultados de los test de Shapiro-Wilk,
ninguna de las variables predictoras tiene una distribución normal, esto
hará que el algoritmo LDA no funcione correctamente. Las varianzas de
las variables es 1 para todas (esto ocurre porque los datos están
normalizados).

    \begin{Verbatim}[commandchars=\\\{\}]
{\color{incolor}In [{\color{incolor}69}]:} \PY{c+c1}{\PYZsh{} Ejecutamos el algoritmo LDA.}
         \PY{n}{temp} \PY{o}{=} \PY{n}{hr.train}
         \PY{n}{temp}\PY{o}{\PYZdl{}}\PY{n}{Class} \PY{o}{=} \PY{n}{hr.labels.train}
         \PY{n}{lda.fit} \PY{o}{=} \PY{n+nf}{lda}\PY{p}{(}\PY{n}{Class}\PY{o}{\PYZti{}}\PY{n}{Age}\PY{o}{+}\PY{n}{EducationalLevel}\PY{o}{+}\PY{n}{MaritalStatus}\PY{p}{,}
                      \PY{n}{data}\PY{o}{=}\PY{n}{temp}\PY{p}{)}
         \PY{n}{lda.fit}
\end{Verbatim}


    
    \begin{verbatim}
Call:
lda(Class ~ Age + EducationalLevel + MaritalStatus, data = temp)

Prior probabilities of groups:
      1       2       3 
0.40625 0.40000 0.19375 

Group means:
         Age EducationalLevel MaritalStatus
1 -0.3669775      -0.33417501   -0.30137255
2 -0.0233205      -0.05663549   -0.08995048
3  0.8176144       0.81761441    0.81761441

Coefficients of linear discriminants:
                      LD1         LD2
Age              0.833050 -0.69799777
EducationalLevel 0.815173  0.00993965
MaritalStatus    0.797296  0.71787707

Proportion of trace:
   LD1    LD2 
0.9985 0.0015 
    \end{verbatim}

    
    \begin{Verbatim}[commandchars=\\\{\}]
{\color{incolor}In [{\color{incolor}70}]:} \PY{n+nf}{require}\PY{p}{(}\PY{n}{klaR}\PY{p}{)}
         \PY{n+nf}{partimat}\PY{p}{(}\PY{n}{Class}\PY{o}{\PYZti{}}\PY{n}{Age}\PY{o}{+}\PY{n}{EducationalLevel}\PY{o}{+}\PY{n}{MaritalStatus}\PY{p}{,}
                      \PY{n}{data}\PY{o}{=}\PY{n}{temp}\PY{p}{,} \PY{n}{method}\PY{o}{=}\PY{l+s}{\PYZdq{}}\PY{l+s}{lda\PYZdq{}}\PY{p}{)}
\end{Verbatim}


    \begin{center}
    \adjustimage{max size={0.9\linewidth}{0.9\paperheight}}{output_52_0.png}
    \end{center}
    { \hspace*{\fill} \\}
    
    \begin{center}
    \adjustimage{max size={0.9\linewidth}{0.9\paperheight}}{output_52_1.png}
    \end{center}
    { \hspace*{\fill} \\}
    
    \begin{Verbatim}[commandchars=\\\{\}]
{\color{incolor}In [{\color{incolor}96}]:} \PY{c+c1}{\PYZsh{} Calculamos el error de LDA.}
         \PY{n}{lda.pred} \PY{o}{=} \PY{n+nf}{predict}\PY{p}{(}\PY{n}{lda.fit}\PY{p}{,} \PY{n}{temp}\PY{n}{[}\PY{p}{,}\PY{l+m}{\PYZhy{}4}\PY{n}{]}\PY{p}{)}
         \PY{n+nf}{mean}\PY{p}{(}\PY{n}{lda.pred}\PY{o}{\PYZdl{}}\PY{n}{class}\PY{o}{!=}\PY{n}{temp}\PY{o}{\PYZdl{}}\PY{n}{Class}\PY{p}{)}
\end{Verbatim}


    0.45625

    
    \begin{Verbatim}[commandchars=\\\{\}]
{\color{incolor}In [{\color{incolor}76}]:} \PY{c+c1}{\PYZsh{} Validación cruzada para LDA.}
         \PY{n}{nombre} \PY{o}{\PYZlt{}\PYZhy{}} \PY{l+s}{\PYZdq{}}\PY{l+s}{hayes\PYZhy{}roth/hayes\PYZhy{}roth\PYZdq{}}
         
         \PY{n}{run\PYZus{}knn\PYZus{}fold} \PY{o}{\PYZlt{}\PYZhy{}} \PY{n+nf}{function}\PY{p}{(}\PY{n}{i}\PY{p}{,} \PY{n}{x}\PY{p}{,} \PY{n}{tt} \PY{o}{=} \PY{l+s}{\PYZdq{}}\PY{l+s}{test\PYZdq{}}\PY{p}{)} \PY{p}{\PYZob{}}
             \PY{n}{file} \PY{o}{\PYZlt{}\PYZhy{}} \PY{n+nf}{paste}\PY{p}{(}\PY{n}{x}\PY{p}{,} \PY{l+s}{\PYZdq{}}\PY{l+s}{\PYZhy{}10\PYZhy{}\PYZdq{}}\PY{p}{,} \PY{n}{i}\PY{p}{,} \PY{l+s}{\PYZdq{}}\PY{l+s}{tra.dat\PYZdq{}}\PY{p}{,} \PY{n}{sep}\PY{o}{=}\PY{l+s}{\PYZdq{}}\PY{l+s}{\PYZdq{}}\PY{p}{)}
             \PY{n}{x\PYZus{}tra} \PY{o}{\PYZlt{}\PYZhy{}} \PY{n+nf}{read.csv}\PY{p}{(}\PY{n}{file}\PY{p}{,} \PY{n}{comment.char}\PY{o}{=}\PY{l+s}{\PYZdq{}}\PY{l+s}{@\PYZdq{}}\PY{p}{,} \PY{n}{header}\PY{o}{=}\PY{k+kc}{FALSE}\PY{p}{)}
             \PY{n}{file} \PY{o}{\PYZlt{}\PYZhy{}} \PY{n+nf}{paste}\PY{p}{(}\PY{n}{x}\PY{p}{,} \PY{l+s}{\PYZdq{}}\PY{l+s}{\PYZhy{}10\PYZhy{}\PYZdq{}}\PY{p}{,} \PY{n}{i}\PY{p}{,} \PY{l+s}{\PYZdq{}}\PY{l+s}{tst.dat\PYZdq{}}\PY{p}{,} \PY{n}{sep}\PY{o}{=}\PY{l+s}{\PYZdq{}}\PY{l+s}{\PYZdq{}}\PY{p}{)}
             \PY{n}{x\PYZus{}tst} \PY{o}{\PYZlt{}\PYZhy{}} \PY{n+nf}{read.csv}\PY{p}{(}\PY{n}{file}\PY{p}{,} \PY{n}{comment.char}\PY{o}{=}\PY{l+s}{\PYZdq{}}\PY{l+s}{@\PYZdq{}}\PY{p}{,} \PY{n}{header}\PY{o}{=}\PY{k+kc}{FALSE}\PY{p}{)}
             \PY{n}{In} \PY{o}{\PYZlt{}\PYZhy{}} \PY{n+nf}{length}\PY{p}{(}\PY{n+nf}{names}\PY{p}{(}\PY{n}{x\PYZus{}tra}\PY{p}{)}\PY{p}{)} \PY{o}{\PYZhy{}} \PY{l+m}{1}
             \PY{n+nf}{names}\PY{p}{(}\PY{n}{x\PYZus{}tra}\PY{p}{)}\PY{n}{[1}\PY{o}{:}\PY{n}{In}\PY{n}{]} \PY{o}{\PYZlt{}\PYZhy{}} \PY{n+nf}{paste }\PY{p}{(}\PY{l+s}{\PYZdq{}}\PY{l+s}{X\PYZdq{}}\PY{p}{,} \PY{l+m}{1}\PY{o}{:}\PY{n}{In}\PY{p}{,} \PY{n}{sep}\PY{o}{=}\PY{l+s}{\PYZdq{}}\PY{l+s}{\PYZdq{}}\PY{p}{)}
             \PY{n+nf}{names}\PY{p}{(}\PY{n}{x\PYZus{}tra}\PY{p}{)}\PY{n}{[In}\PY{l+m}{+1}\PY{n}{]} \PY{o}{\PYZlt{}\PYZhy{}} \PY{l+s}{\PYZdq{}}\PY{l+s}{Y\PYZdq{}}
             \PY{n+nf}{names}\PY{p}{(}\PY{n}{x\PYZus{}tst}\PY{p}{)}\PY{n}{[1}\PY{o}{:}\PY{n}{In}\PY{n}{]} \PY{o}{\PYZlt{}\PYZhy{}} \PY{n+nf}{paste }\PY{p}{(}\PY{l+s}{\PYZdq{}}\PY{l+s}{X\PYZdq{}}\PY{p}{,} \PY{l+m}{1}\PY{o}{:}\PY{n}{In}\PY{p}{,} \PY{n}{sep}\PY{o}{=}\PY{l+s}{\PYZdq{}}\PY{l+s}{\PYZdq{}}\PY{p}{)}
             \PY{n+nf}{names}\PY{p}{(}\PY{n}{x\PYZus{}tst}\PY{p}{)}\PY{n}{[In}\PY{l+m}{+1}\PY{n}{]} \PY{o}{\PYZlt{}\PYZhy{}} \PY{l+s}{\PYZdq{}}\PY{l+s}{Y\PYZdq{}}
             \PY{n+nf}{if }\PY{p}{(}\PY{n}{tt} \PY{o}{==} \PY{l+s}{\PYZdq{}}\PY{l+s}{train\PYZdq{}}\PY{p}{)} \PY{p}{\PYZob{}}
                 \PY{n}{test} \PY{o}{\PYZlt{}\PYZhy{}} \PY{n}{x\PYZus{}tra}
             \PY{p}{\PYZcb{}}
             \PY{n}{else} \PY{p}{\PYZob{}}
                 \PY{n}{test} \PY{o}{\PYZlt{}\PYZhy{}} \PY{n}{x\PYZus{}tst}
             \PY{p}{\PYZcb{}}
             
             \PY{n}{hr.train} \PY{o}{=} \PY{n}{x\PYZus{}tra}\PY{n}{[}\PY{p}{,}\PY{n+nf}{c}\PY{p}{(}\PY{l+s}{\PYZdq{}}\PY{l+s}{X2\PYZdq{}}\PY{p}{,}\PY{l+s}{\PYZdq{}}\PY{l+s}{X3\PYZdq{}}\PY{p}{,}\PY{l+s}{\PYZdq{}}\PY{l+s}{X4\PYZdq{}}\PY{p}{)}\PY{n}{]}
             \PY{n}{hr.labels.train} \PY{o}{=} \PY{n}{x\PYZus{}tra}\PY{n}{[}\PY{p}{,}\PY{l+s}{\PYZdq{}}\PY{l+s}{Y\PYZdq{}}\PY{n}{]}
             \PY{n}{hr.labels.train} \PY{o}{=} \PY{n+nf}{factor}\PY{p}{(}\PY{n}{hr.labels.train}\PY{p}{,} \PY{n}{levels}\PY{o}{=}\PY{n+nf}{c}\PY{p}{(}\PY{l+m}{1}\PY{p}{,}\PY{l+m}{2}\PY{p}{,}\PY{l+m}{3}\PY{p}{)}\PY{p}{)}
             \PY{n}{hr.train}\PY{o}{\PYZdl{}}\PY{n}{Y} \PY{o}{=} \PY{n}{hr.labels.train}
             \PY{n}{fitMulti}\PY{o}{=}\PY{n+nf}{lda}\PY{p}{(}\PY{n}{Y}\PY{o}{\PYZti{}}\PY{n}{X2}\PY{o}{+}\PY{n}{X3}\PY{o}{+}\PY{n}{X4}\PY{p}{,} \PY{n}{data}\PY{o}{=}\PY{n}{hr.train}\PY{p}{)}
             \PY{n}{labels} \PY{o}{=} \PY{n+nf}{factor}\PY{p}{(}\PY{n}{test}\PY{n}{[}\PY{p}{,}\PY{l+s}{\PYZdq{}}\PY{l+s}{Y\PYZdq{}}\PY{n}{]}\PY{p}{,}\PY{n}{levels}\PY{o}{=}\PY{n+nf}{c}\PY{p}{(}\PY{l+m}{1}\PY{p}{,}\PY{l+m}{2}\PY{p}{,}\PY{l+m}{3}\PY{p}{)}\PY{p}{)}
             \PY{n}{yprime}\PY{o}{=}\PY{n+nf}{predict}\PY{p}{(}\PY{n}{fitMulti}\PY{p}{,}\PY{n}{test}\PY{n}{[}\PY{p}{,}\PY{n+nf}{c}\PY{p}{(}\PY{l+s}{\PYZdq{}}\PY{l+s}{X2\PYZdq{}}\PY{p}{,}\PY{l+s}{\PYZdq{}}\PY{l+s}{X3\PYZdq{}}\PY{p}{,}\PY{l+s}{\PYZdq{}}\PY{l+s}{X4\PYZdq{}}\PY{p}{)}\PY{n}{]}\PY{p}{)}
             \PY{n}{err} \PY{o}{=} \PY{n+nf}{mean}\PY{p}{(}\PY{n}{yprime}\PY{o}{\PYZdl{}}\PY{n}{class}\PY{o}{!=}\PY{n}{labels}\PY{p}{)}
             
         \PY{p}{\PYZcb{}}
         \PY{n}{ldaERRtrain.all}\PY{o}{\PYZlt{}\PYZhy{}}\PY{n+nf}{mean}\PY{p}{(}\PY{n+nf}{sapply}\PY{p}{(}\PY{l+m}{1}\PY{o}{:}\PY{l+m}{10}\PY{p}{,}\PY{n}{run\PYZus{}knn\PYZus{}fold}\PY{p}{,}\PY{n}{nombre}\PY{p}{,}\PY{l+s}{\PYZdq{}}\PY{l+s}{train\PYZdq{}}\PY{p}{)}\PY{p}{)}
         \PY{n}{ldaERRtest.all}\PY{o}{\PYZlt{}\PYZhy{}}\PY{n+nf}{mean}\PY{p}{(}\PY{n+nf}{sapply}\PY{p}{(}\PY{l+m}{1}\PY{o}{:}\PY{l+m}{10}\PY{p}{,}\PY{n}{run\PYZus{}knn\PYZus{}fold}\PY{p}{,}\PY{n}{nombre}\PY{p}{,}\PY{l+s}{\PYZdq{}}\PY{l+s}{test\PYZdq{}}\PY{p}{)}\PY{p}{)}
         \PY{n+nf}{print}\PY{p}{(}\PY{n}{ldaERRtrain.all}\PY{p}{)}
         \PY{n+nf}{print}\PY{p}{(}\PY{n}{ldaERRtest.all}\PY{p}{)}
\end{Verbatim}


    \begin{Verbatim}[commandchars=\\\{\}]
[1] 0.4583333
[1] 0.4625

    \end{Verbatim}

    Como se puede ver, el algoritmo LDA no tiene sobreaprendizaje, pero sí
que obtiene peores resultados que el algoritmo KNN.

    \hypertarget{modelos-con-algoritmo-qda}{%
\subsubsection{Modelos con algoritmo
QDA}\label{modelos-con-algoritmo-qda}}

En este apartado utilizaremos el algoritmo QDA para predecir las clases,
para ello utilizaremos comprobaremos sí los datos cumplen las
características necesarias para que QDA funcione correctamente. Tras
esto, crearemos un modelo con las variables \emph{Age},
\emph{MaritalStatus} y \emph{EducationalLevel} y realizaremos validación
cruzada.

    \begin{Verbatim}[commandchars=\\\{\}]
{\color{incolor}In [{\color{incolor}93}]:} \PY{c+c1}{\PYZsh{} Comprobamos que las varianzas de los predictores para las diferentes clases}
         \PY{c+c1}{\PYZsh{} son diferentes.}
         \PY{n+nf}{cat}\PY{p}{(}\PY{l+s}{\PYZdq{}}\PY{l+s}{Varianzas para Age:\PYZbs{}n\PYZdq{}}\PY{p}{)}
         \PY{n+nf}{var}\PY{p}{(}\PY{n}{temp}\PY{n}{[temp}\PY{o}{\PYZdl{}}\PY{n}{Class}\PY{o}{==}\PY{l+m}{1}\PY{p}{,}\PY{n}{]}\PY{o}{\PYZdl{}}\PY{n}{Age}\PY{p}{)}
         \PY{n+nf}{var}\PY{p}{(}\PY{n}{temp}\PY{n}{[temp}\PY{o}{\PYZdl{}}\PY{n}{Class}\PY{o}{==}\PY{l+m}{2}\PY{p}{,}\PY{n}{]}\PY{o}{\PYZdl{}}\PY{n}{Age}\PY{p}{)}
         \PY{n+nf}{var}\PY{p}{(}\PY{n}{temp}\PY{n}{[temp}\PY{o}{\PYZdl{}}\PY{n}{Class}\PY{o}{==}\PY{l+m}{3}\PY{p}{,}\PY{n}{]}\PY{o}{\PYZdl{}}\PY{n}{Age}\PY{p}{)}
         \PY{n+nf}{cat}\PY{p}{(}\PY{l+s}{\PYZdq{}}\PY{l+s}{Varianzas para EducationalLevel:\PYZbs{}n\PYZdq{}}\PY{p}{)}
         \PY{n+nf}{var}\PY{p}{(}\PY{n}{temp}\PY{n}{[temp}\PY{o}{\PYZdl{}}\PY{n}{Class}\PY{o}{==}\PY{l+m}{1}\PY{p}{,}\PY{n}{]}\PY{o}{\PYZdl{}}\PY{n}{EducationalLevel}\PY{p}{)}
         \PY{n+nf}{var}\PY{p}{(}\PY{n}{temp}\PY{n}{[temp}\PY{o}{\PYZdl{}}\PY{n}{Class}\PY{o}{==}\PY{l+m}{2}\PY{p}{,}\PY{n}{]}\PY{o}{\PYZdl{}}\PY{n}{EducationalLevel}\PY{p}{)}
         \PY{n+nf}{var}\PY{p}{(}\PY{n}{temp}\PY{n}{[temp}\PY{o}{\PYZdl{}}\PY{n}{Class}\PY{o}{==}\PY{l+m}{3}\PY{p}{,}\PY{n}{]}\PY{o}{\PYZdl{}}\PY{n}{EducationalLevel}\PY{p}{)}
         \PY{n+nf}{cat}\PY{p}{(}\PY{l+s}{\PYZdq{}}\PY{l+s}{Varianzas para MaritalStatus:\PYZbs{}n\PYZdq{}}\PY{p}{)}
         \PY{n+nf}{var}\PY{p}{(}\PY{n}{temp}\PY{n}{[temp}\PY{o}{\PYZdl{}}\PY{n}{Class}\PY{o}{==}\PY{l+m}{1}\PY{p}{,}\PY{n}{]}\PY{o}{\PYZdl{}}\PY{n}{MaritalStatus}\PY{p}{)}
         \PY{n+nf}{var}\PY{p}{(}\PY{n}{temp}\PY{n}{[temp}\PY{o}{\PYZdl{}}\PY{n}{Class}\PY{o}{==}\PY{l+m}{2}\PY{p}{,}\PY{n}{]}\PY{o}{\PYZdl{}}\PY{n}{MaritalStatus}\PY{p}{)}
         \PY{n+nf}{var}\PY{p}{(}\PY{n}{temp}\PY{n}{[temp}\PY{o}{\PYZdl{}}\PY{n}{Class}\PY{o}{==}\PY{l+m}{3}\PY{p}{,}\PY{n}{]}\PY{o}{\PYZdl{}}\PY{n}{MaritalStatus}\PY{p}{)}
\end{Verbatim}


    \begin{Verbatim}[commandchars=\\\{\}]
Varianzas para Age:

    \end{Verbatim}

    0.695029416616264

    
    0.520625914769053

    
    1.74022919596947

    
    \begin{Verbatim}[commandchars=\\\{\}]
Varianzas para EducationalLevel:

    \end{Verbatim}

    0.720164128223456

    
    0.516115890602131

    
    1.74022919596947

    
    \begin{Verbatim}[commandchars=\\\{\}]
Varianzas para MaritalStatus:

    \end{Verbatim}

    0.743113212734371

    
    0.509350854351748

    
    1.74022919596947

    
    Como todas las varianzas son diferentes, en principio QDA debería de
funcionar mejor que LDA. Lo siguiente que haremos será crear un modelo y
ver los resultados que obtiene.

    \begin{Verbatim}[commandchars=\\\{\}]
{\color{incolor}In [{\color{incolor}94}]:} \PY{c+c1}{\PYZsh{} creamos el modelo con QDA.}
         \PY{n}{qda.fit} \PY{o}{=} \PY{n+nf}{qda}\PY{p}{(}\PY{n}{Class}\PY{o}{\PYZti{}}\PY{n}{Age}\PY{o}{+}\PY{n}{EducationalLevel}\PY{o}{+}\PY{n}{MaritalStatus}\PY{p}{,}
                      \PY{n}{data}\PY{o}{=}\PY{n}{temp}\PY{p}{)}
         \PY{n}{qda.fit}
\end{Verbatim}


    
    \begin{verbatim}
Call:
qda(Class ~ Age + EducationalLevel + MaritalStatus, data = temp)

Prior probabilities of groups:
      1       2       3 
0.40625 0.40000 0.19375 

Group means:
         Age EducationalLevel MaritalStatus
1 -0.3669775      -0.33417501   -0.30137255
2 -0.0233205      -0.05663549   -0.08995048
3  0.8176144       0.81761441    0.81761441
    \end{verbatim}

    
    \begin{Verbatim}[commandchars=\\\{\}]
{\color{incolor}In [{\color{incolor}95}]:} \PY{c+c1}{\PYZsh{} dibujamos los resultados obtenidos por QDA.}
         \PY{n+nf}{partimat}\PY{p}{(}\PY{n}{Class}\PY{o}{\PYZti{}}\PY{n}{Age}\PY{o}{+}\PY{n}{EducationalLevel}\PY{o}{+}\PY{n}{MaritalStatus}\PY{p}{,}
                      \PY{n}{data}\PY{o}{=}\PY{n}{temp}\PY{p}{,} \PY{n}{method}\PY{o}{=}\PY{l+s}{\PYZdq{}}\PY{l+s}{qda\PYZdq{}}\PY{p}{)}
\end{Verbatim}


    \begin{center}
    \adjustimage{max size={0.9\linewidth}{0.9\paperheight}}{output_60_0.png}
    \end{center}
    { \hspace*{\fill} \\}
    
    \begin{center}
    \adjustimage{max size={0.9\linewidth}{0.9\paperheight}}{output_60_1.png}
    \end{center}
    { \hspace*{\fill} \\}
    
    \begin{Verbatim}[commandchars=\\\{\}]
{\color{incolor}In [{\color{incolor}98}]:} \PY{c+c1}{\PYZsh{} Calculamos el error para qda.}
         \PY{n}{qda.pred} \PY{o}{=} \PY{n+nf}{predict}\PY{p}{(}\PY{n}{qda.fit}\PY{p}{,} \PY{n}{temp}\PY{n}{[}\PY{p}{,}\PY{l+m}{\PYZhy{}4}\PY{n}{]}\PY{p}{)}
         \PY{n+nf}{mean}\PY{p}{(}\PY{n}{qda.pred}\PY{o}{\PYZdl{}}\PY{n}{class}\PY{o}{!=}\PY{n}{temp}\PY{o}{\PYZdl{}}\PY{n}{Class}\PY{p}{)}
\end{Verbatim}


    0.275

    
    \begin{Verbatim}[commandchars=\\\{\}]
{\color{incolor}In [{\color{incolor}100}]:} \PY{c+c1}{\PYZsh{} Realizamos validación cruzada para QDA}
          \PY{n}{nombre} \PY{o}{\PYZlt{}\PYZhy{}} \PY{l+s}{\PYZdq{}}\PY{l+s}{hayes\PYZhy{}roth/hayes\PYZhy{}roth\PYZdq{}}
          
          \PY{n}{run\PYZus{}knn\PYZus{}fold} \PY{o}{\PYZlt{}\PYZhy{}} \PY{n+nf}{function}\PY{p}{(}\PY{n}{i}\PY{p}{,} \PY{n}{x}\PY{p}{,} \PY{n}{tt} \PY{o}{=} \PY{l+s}{\PYZdq{}}\PY{l+s}{test\PYZdq{}}\PY{p}{)} \PY{p}{\PYZob{}}
              \PY{n}{file} \PY{o}{\PYZlt{}\PYZhy{}} \PY{n+nf}{paste}\PY{p}{(}\PY{n}{x}\PY{p}{,} \PY{l+s}{\PYZdq{}}\PY{l+s}{\PYZhy{}10\PYZhy{}\PYZdq{}}\PY{p}{,} \PY{n}{i}\PY{p}{,} \PY{l+s}{\PYZdq{}}\PY{l+s}{tra.dat\PYZdq{}}\PY{p}{,} \PY{n}{sep}\PY{o}{=}\PY{l+s}{\PYZdq{}}\PY{l+s}{\PYZdq{}}\PY{p}{)}
              \PY{n}{x\PYZus{}tra} \PY{o}{\PYZlt{}\PYZhy{}} \PY{n+nf}{read.csv}\PY{p}{(}\PY{n}{file}\PY{p}{,} \PY{n}{comment.char}\PY{o}{=}\PY{l+s}{\PYZdq{}}\PY{l+s}{@\PYZdq{}}\PY{p}{,} \PY{n}{header}\PY{o}{=}\PY{k+kc}{FALSE}\PY{p}{)}
              \PY{n}{file} \PY{o}{\PYZlt{}\PYZhy{}} \PY{n+nf}{paste}\PY{p}{(}\PY{n}{x}\PY{p}{,} \PY{l+s}{\PYZdq{}}\PY{l+s}{\PYZhy{}10\PYZhy{}\PYZdq{}}\PY{p}{,} \PY{n}{i}\PY{p}{,} \PY{l+s}{\PYZdq{}}\PY{l+s}{tst.dat\PYZdq{}}\PY{p}{,} \PY{n}{sep}\PY{o}{=}\PY{l+s}{\PYZdq{}}\PY{l+s}{\PYZdq{}}\PY{p}{)}
              \PY{n}{x\PYZus{}tst} \PY{o}{\PYZlt{}\PYZhy{}} \PY{n+nf}{read.csv}\PY{p}{(}\PY{n}{file}\PY{p}{,} \PY{n}{comment.char}\PY{o}{=}\PY{l+s}{\PYZdq{}}\PY{l+s}{@\PYZdq{}}\PY{p}{,} \PY{n}{header}\PY{o}{=}\PY{k+kc}{FALSE}\PY{p}{)}
              \PY{n}{In} \PY{o}{\PYZlt{}\PYZhy{}} \PY{n+nf}{length}\PY{p}{(}\PY{n+nf}{names}\PY{p}{(}\PY{n}{x\PYZus{}tra}\PY{p}{)}\PY{p}{)} \PY{o}{\PYZhy{}} \PY{l+m}{1}
              \PY{n+nf}{names}\PY{p}{(}\PY{n}{x\PYZus{}tra}\PY{p}{)}\PY{n}{[1}\PY{o}{:}\PY{n}{In}\PY{n}{]} \PY{o}{\PYZlt{}\PYZhy{}} \PY{n+nf}{paste }\PY{p}{(}\PY{l+s}{\PYZdq{}}\PY{l+s}{X\PYZdq{}}\PY{p}{,} \PY{l+m}{1}\PY{o}{:}\PY{n}{In}\PY{p}{,} \PY{n}{sep}\PY{o}{=}\PY{l+s}{\PYZdq{}}\PY{l+s}{\PYZdq{}}\PY{p}{)}
              \PY{n+nf}{names}\PY{p}{(}\PY{n}{x\PYZus{}tra}\PY{p}{)}\PY{n}{[In}\PY{l+m}{+1}\PY{n}{]} \PY{o}{\PYZlt{}\PYZhy{}} \PY{l+s}{\PYZdq{}}\PY{l+s}{Y\PYZdq{}}
              \PY{n+nf}{names}\PY{p}{(}\PY{n}{x\PYZus{}tst}\PY{p}{)}\PY{n}{[1}\PY{o}{:}\PY{n}{In}\PY{n}{]} \PY{o}{\PYZlt{}\PYZhy{}} \PY{n+nf}{paste }\PY{p}{(}\PY{l+s}{\PYZdq{}}\PY{l+s}{X\PYZdq{}}\PY{p}{,} \PY{l+m}{1}\PY{o}{:}\PY{n}{In}\PY{p}{,} \PY{n}{sep}\PY{o}{=}\PY{l+s}{\PYZdq{}}\PY{l+s}{\PYZdq{}}\PY{p}{)}
              \PY{n+nf}{names}\PY{p}{(}\PY{n}{x\PYZus{}tst}\PY{p}{)}\PY{n}{[In}\PY{l+m}{+1}\PY{n}{]} \PY{o}{\PYZlt{}\PYZhy{}} \PY{l+s}{\PYZdq{}}\PY{l+s}{Y\PYZdq{}}
              \PY{n+nf}{if }\PY{p}{(}\PY{n}{tt} \PY{o}{==} \PY{l+s}{\PYZdq{}}\PY{l+s}{train\PYZdq{}}\PY{p}{)} \PY{p}{\PYZob{}}
                  \PY{n}{test} \PY{o}{\PYZlt{}\PYZhy{}} \PY{n}{x\PYZus{}tra}
              \PY{p}{\PYZcb{}}
              \PY{n}{else} \PY{p}{\PYZob{}}
                  \PY{n}{test} \PY{o}{\PYZlt{}\PYZhy{}} \PY{n}{x\PYZus{}tst}
              \PY{p}{\PYZcb{}}
              
              \PY{n}{hr.train} \PY{o}{=} \PY{n}{x\PYZus{}tra}\PY{n}{[}\PY{p}{,}\PY{n+nf}{c}\PY{p}{(}\PY{l+s}{\PYZdq{}}\PY{l+s}{X2\PYZdq{}}\PY{p}{,}\PY{l+s}{\PYZdq{}}\PY{l+s}{X3\PYZdq{}}\PY{p}{,}\PY{l+s}{\PYZdq{}}\PY{l+s}{X4\PYZdq{}}\PY{p}{)}\PY{n}{]}
              \PY{n}{hr.labels.train} \PY{o}{=} \PY{n}{x\PYZus{}tra}\PY{n}{[}\PY{p}{,}\PY{l+s}{\PYZdq{}}\PY{l+s}{Y\PYZdq{}}\PY{n}{]}
              \PY{n}{hr.labels.train} \PY{o}{=} \PY{n+nf}{factor}\PY{p}{(}\PY{n}{hr.labels.train}\PY{p}{,} \PY{n}{levels}\PY{o}{=}\PY{n+nf}{c}\PY{p}{(}\PY{l+m}{1}\PY{p}{,}\PY{l+m}{2}\PY{p}{,}\PY{l+m}{3}\PY{p}{)}\PY{p}{)}
              \PY{n}{hr.train}\PY{o}{\PYZdl{}}\PY{n}{Y} \PY{o}{=} \PY{n}{hr.labels.train}
              \PY{n}{fitMulti}\PY{o}{=}\PY{n+nf}{qda}\PY{p}{(}\PY{n}{Y}\PY{o}{\PYZti{}}\PY{n}{X2}\PY{o}{+}\PY{n}{X3}\PY{o}{+}\PY{n}{X4}\PY{p}{,} \PY{n}{data}\PY{o}{=}\PY{n}{hr.train}\PY{p}{)}
              \PY{n}{labels} \PY{o}{=} \PY{n+nf}{factor}\PY{p}{(}\PY{n}{test}\PY{n}{[}\PY{p}{,}\PY{l+s}{\PYZdq{}}\PY{l+s}{Y\PYZdq{}}\PY{n}{]}\PY{p}{,}\PY{n}{levels}\PY{o}{=}\PY{n+nf}{c}\PY{p}{(}\PY{l+m}{1}\PY{p}{,}\PY{l+m}{2}\PY{p}{,}\PY{l+m}{3}\PY{p}{)}\PY{p}{)}
              \PY{n}{yprime}\PY{o}{=}\PY{n+nf}{predict}\PY{p}{(}\PY{n}{fitMulti}\PY{p}{,}\PY{n}{test}\PY{n}{[}\PY{p}{,}\PY{n+nf}{c}\PY{p}{(}\PY{l+s}{\PYZdq{}}\PY{l+s}{X2\PYZdq{}}\PY{p}{,}\PY{l+s}{\PYZdq{}}\PY{l+s}{X3\PYZdq{}}\PY{p}{,}\PY{l+s}{\PYZdq{}}\PY{l+s}{X4\PYZdq{}}\PY{p}{)}\PY{n}{]}\PY{p}{)}
              \PY{n}{err} \PY{o}{=} \PY{n+nf}{mean}\PY{p}{(}\PY{n}{yprime}\PY{o}{\PYZdl{}}\PY{n}{class}\PY{o}{!=}\PY{n}{labels}\PY{p}{)}
              
          \PY{p}{\PYZcb{}}
          \PY{n}{qdaERRtrain.all}\PY{o}{\PYZlt{}\PYZhy{}}\PY{n+nf}{mean}\PY{p}{(}\PY{n+nf}{sapply}\PY{p}{(}\PY{l+m}{1}\PY{o}{:}\PY{l+m}{10}\PY{p}{,}\PY{n}{run\PYZus{}knn\PYZus{}fold}\PY{p}{,}\PY{n}{nombre}\PY{p}{,}\PY{l+s}{\PYZdq{}}\PY{l+s}{train\PYZdq{}}\PY{p}{)}\PY{p}{)}
          \PY{n}{qdaERRtest.all}\PY{o}{\PYZlt{}\PYZhy{}}\PY{n+nf}{mean}\PY{p}{(}\PY{n+nf}{sapply}\PY{p}{(}\PY{l+m}{1}\PY{o}{:}\PY{l+m}{10}\PY{p}{,}\PY{n}{run\PYZus{}knn\PYZus{}fold}\PY{p}{,}\PY{n}{nombre}\PY{p}{,}\PY{l+s}{\PYZdq{}}\PY{l+s}{test\PYZdq{}}\PY{p}{)}\PY{p}{)}
          \PY{n+nf}{print}\PY{p}{(}\PY{n}{qdaERRtrain.all}\PY{p}{)}
          \PY{n+nf}{print}\PY{p}{(}\PY{n}{qdaERRtest.all}\PY{p}{)}
\end{Verbatim}


    \begin{Verbatim}[commandchars=\\\{\}]
[1] 0.2590278
[1] 0.35

    \end{Verbatim}

    Como se puede ver, QDA obtiene mejores resultados que LDA, pero tiene
sobreaprendizaje. Por ahora, el mejor modelo que hemos obtenido es el
generado por KNN, ya que obtiene un entre 22-26 por ciento de error, lo
cual mejora en un 10\% a QDA y un 20\% a LDA.

    \hypertarget{modelos-con-uxe1rboles-de-decisiuxf3n}{%
\subsubsection{Modelos con árboles de
decisión}\label{modelos-con-uxe1rboles-de-decisiuxf3n}}

Este apartado es adicional y se ha hecho ya que parece que el dataset
puede predecirse mejor con un árbol de decisión, ya que existen muy
pocas diferencias entre los datos de cada clase y cada variable (pueden
verse los scatterplots divididos por clase en el apartado de análisis
exploratorio).

    \begin{Verbatim}[commandchars=\\\{\}]
{\color{incolor}In [{\color{incolor}115}]:} \PY{c+c1}{\PYZsh{} cargamos la librería que contiene los árboles de decisión.}
          \PY{n+nf}{require}\PY{p}{(}\PY{n}{tree}\PY{p}{)}
          
          \PY{n}{temp}\PY{o}{=}\PY{n}{hayesroth}
          \PY{n}{temp}\PY{o}{\PYZdl{}}\PY{n}{Class} \PY{o}{=} \PY{n+nf}{factor}\PY{p}{(}\PY{n}{temp}\PY{o}{\PYZdl{}}\PY{n}{Class}\PY{p}{,}\PY{n}{levels}\PY{o}{=}\PY{n+nf}{c}\PY{p}{(}\PY{l+m}{1}\PY{p}{,}\PY{l+m}{2}\PY{p}{,}\PY{l+m}{3}\PY{p}{)}\PY{p}{)}
          
          \PY{n+nf}{set.seed }\PY{p}{(}\PY{l+m}{2}\PY{p}{)}
          \PY{n}{train}\PY{o}{=}\PY{n+nf}{sample }\PY{p}{(}\PY{l+m}{1}\PY{o}{:}\PY{n+nf}{nrow}\PY{p}{(}\PY{n}{temp}\PY{p}{)}\PY{p}{,} \PY{n+nf}{round}\PY{p}{(}\PY{n+nf}{nrow}\PY{p}{(}\PY{n}{temp}\PY{p}{)}\PY{o}{*}\PY{l+m}{0.8}\PY{p}{)} \PY{p}{)}
          \PY{n}{hayes.test}\PY{o}{=}\PY{n}{temp}\PY{n}{[}\PY{o}{\PYZhy{}}\PY{n}{train} \PY{p}{,}\PY{n}{]}
          
          \PY{c+c1}{\PYZsh{} Construyo el arbol sobre el conjunto de entrenamiento}
          \PY{n}{tree.hayes} \PY{o}{=}\PY{n+nf}{tree}\PY{p}{(}\PY{n}{Class}\PY{o}{\PYZti{}}\PY{n}{Age}\PY{o}{+}\PY{n}{EducationalLevel}\PY{o}{+}\PY{n}{MaritalStatus} \PY{p}{,}
                          \PY{n}{temp} \PY{p}{,}\PY{n}{subset} \PY{o}{=}\PY{n}{train} \PY{p}{)}
          
          \PY{n+nf}{summary}\PY{p}{(}\PY{n}{tree.hayes}\PY{p}{)}
          
          \PY{c+c1}{\PYZsh{} Aplico el arbol sobre el conjunto de test}
          \PY{n}{tree.pred} \PY{o}{=}\PY{n+nf}{predict }\PY{p}{(}\PY{n}{tree.hayes}\PY{p}{,}\PY{n}{hayes.test}\PY{p}{,}\PY{n}{type} \PY{o}{=}\PY{l+s}{\PYZdq{}}\PY{l+s}{class\PYZdq{}}\PY{p}{)}
          
          \PY{c+c1}{\PYZsh{} Visualizo la matriz de confusion}
          \PY{n+nf}{table}\PY{p}{(}\PY{n}{tree.pred} \PY{p}{,} \PY{n}{hayes.test}\PY{n}{[}\PY{p}{,}\PY{l+s}{\PYZdq{}}\PY{l+s}{Class\PYZdq{}}\PY{n}{]}\PY{p}{)}
          \PY{n+nf}{mean}\PY{p}{(}\PY{n}{tree.pred}\PY{o}{!=}\PY{n}{hayes.test}\PY{o}{\PYZdl{}}\PY{n}{Class}\PY{p}{)}
\end{Verbatim}


    
    \begin{verbatim}

Classification tree:
tree(formula = Class ~ Age + EducationalLevel + MaritalStatus, 
    data = temp, subset = train)
Number of terminal nodes:  12 
Residual mean deviance:  0.6289 = 72.95 / 116 
Misclassification error rate: 0.1484 = 19 / 128 
    \end{verbatim}

    
    
    \begin{verbatim}
         
tree.pred  1  2  3
        1 12  0  2
        2  4  9  2
        3  0  0  3
    \end{verbatim}

    
    0.25

    
    \begin{Verbatim}[commandchars=\\\{\}]
{\color{incolor}In [{\color{incolor}116}]:} \PY{c+c1}{\PYZsh{} Dibujamos el árbol.}
          \PY{n+nf}{plot}\PY{p}{(}\PY{n}{tree.hayes}\PY{p}{)}
          \PY{n+nf}{text}\PY{p}{(}\PY{n}{tree.hayes}\PY{p}{,} \PY{n}{pretty}\PY{o}{=}\PY{l+m}{0}\PY{p}{)}
\end{Verbatim}


    \begin{center}
    \adjustimage{max size={0.9\linewidth}{0.9\paperheight}}{output_66_0.png}
    \end{center}
    { \hspace*{\fill} \\}
    
    \begin{Verbatim}[commandchars=\\\{\}]
{\color{incolor}In [{\color{incolor}119}]:} \PY{c+c1}{\PYZsh{} Validación cruzada para el árbol de decisión.}
          \PY{n}{nombre} \PY{o}{\PYZlt{}\PYZhy{}} \PY{l+s}{\PYZdq{}}\PY{l+s}{hayes\PYZhy{}roth/hayes\PYZhy{}roth\PYZdq{}}
          
          \PY{n}{run\PYZus{}knn\PYZus{}fold} \PY{o}{\PYZlt{}\PYZhy{}} \PY{n+nf}{function}\PY{p}{(}\PY{n}{i}\PY{p}{,} \PY{n}{x}\PY{p}{,} \PY{n}{tt} \PY{o}{=} \PY{l+s}{\PYZdq{}}\PY{l+s}{test\PYZdq{}}\PY{p}{)} \PY{p}{\PYZob{}}
              \PY{n}{file} \PY{o}{\PYZlt{}\PYZhy{}} \PY{n+nf}{paste}\PY{p}{(}\PY{n}{x}\PY{p}{,} \PY{l+s}{\PYZdq{}}\PY{l+s}{\PYZhy{}10\PYZhy{}\PYZdq{}}\PY{p}{,} \PY{n}{i}\PY{p}{,} \PY{l+s}{\PYZdq{}}\PY{l+s}{tra.dat\PYZdq{}}\PY{p}{,} \PY{n}{sep}\PY{o}{=}\PY{l+s}{\PYZdq{}}\PY{l+s}{\PYZdq{}}\PY{p}{)}
              \PY{n}{x\PYZus{}tra} \PY{o}{\PYZlt{}\PYZhy{}} \PY{n+nf}{read.csv}\PY{p}{(}\PY{n}{file}\PY{p}{,} \PY{n}{comment.char}\PY{o}{=}\PY{l+s}{\PYZdq{}}\PY{l+s}{@\PYZdq{}}\PY{p}{,} \PY{n}{header}\PY{o}{=}\PY{k+kc}{FALSE}\PY{p}{)}
              \PY{n}{file} \PY{o}{\PYZlt{}\PYZhy{}} \PY{n+nf}{paste}\PY{p}{(}\PY{n}{x}\PY{p}{,} \PY{l+s}{\PYZdq{}}\PY{l+s}{\PYZhy{}10\PYZhy{}\PYZdq{}}\PY{p}{,} \PY{n}{i}\PY{p}{,} \PY{l+s}{\PYZdq{}}\PY{l+s}{tst.dat\PYZdq{}}\PY{p}{,} \PY{n}{sep}\PY{o}{=}\PY{l+s}{\PYZdq{}}\PY{l+s}{\PYZdq{}}\PY{p}{)}
              \PY{n}{x\PYZus{}tst} \PY{o}{\PYZlt{}\PYZhy{}} \PY{n+nf}{read.csv}\PY{p}{(}\PY{n}{file}\PY{p}{,} \PY{n}{comment.char}\PY{o}{=}\PY{l+s}{\PYZdq{}}\PY{l+s}{@\PYZdq{}}\PY{p}{,} \PY{n}{header}\PY{o}{=}\PY{k+kc}{FALSE}\PY{p}{)}
              \PY{n}{In} \PY{o}{\PYZlt{}\PYZhy{}} \PY{n+nf}{length}\PY{p}{(}\PY{n+nf}{names}\PY{p}{(}\PY{n}{x\PYZus{}tra}\PY{p}{)}\PY{p}{)} \PY{o}{\PYZhy{}} \PY{l+m}{1}
              \PY{n+nf}{names}\PY{p}{(}\PY{n}{x\PYZus{}tra}\PY{p}{)}\PY{n}{[1}\PY{o}{:}\PY{n}{In}\PY{n}{]} \PY{o}{\PYZlt{}\PYZhy{}} \PY{n+nf}{paste }\PY{p}{(}\PY{l+s}{\PYZdq{}}\PY{l+s}{X\PYZdq{}}\PY{p}{,} \PY{l+m}{1}\PY{o}{:}\PY{n}{In}\PY{p}{,} \PY{n}{sep}\PY{o}{=}\PY{l+s}{\PYZdq{}}\PY{l+s}{\PYZdq{}}\PY{p}{)}
              \PY{n+nf}{names}\PY{p}{(}\PY{n}{x\PYZus{}tra}\PY{p}{)}\PY{n}{[In}\PY{l+m}{+1}\PY{n}{]} \PY{o}{\PYZlt{}\PYZhy{}} \PY{l+s}{\PYZdq{}}\PY{l+s}{Y\PYZdq{}}
              \PY{n+nf}{names}\PY{p}{(}\PY{n}{x\PYZus{}tst}\PY{p}{)}\PY{n}{[1}\PY{o}{:}\PY{n}{In}\PY{n}{]} \PY{o}{\PYZlt{}\PYZhy{}} \PY{n+nf}{paste }\PY{p}{(}\PY{l+s}{\PYZdq{}}\PY{l+s}{X\PYZdq{}}\PY{p}{,} \PY{l+m}{1}\PY{o}{:}\PY{n}{In}\PY{p}{,} \PY{n}{sep}\PY{o}{=}\PY{l+s}{\PYZdq{}}\PY{l+s}{\PYZdq{}}\PY{p}{)}
              \PY{n+nf}{names}\PY{p}{(}\PY{n}{x\PYZus{}tst}\PY{p}{)}\PY{n}{[In}\PY{l+m}{+1}\PY{n}{]} \PY{o}{\PYZlt{}\PYZhy{}} \PY{l+s}{\PYZdq{}}\PY{l+s}{Y\PYZdq{}}
              \PY{n+nf}{if }\PY{p}{(}\PY{n}{tt} \PY{o}{==} \PY{l+s}{\PYZdq{}}\PY{l+s}{train\PYZdq{}}\PY{p}{)} \PY{p}{\PYZob{}}
                  \PY{n}{test} \PY{o}{\PYZlt{}\PYZhy{}} \PY{n}{x\PYZus{}tra}
              \PY{p}{\PYZcb{}}
              \PY{n}{else} \PY{p}{\PYZob{}}
                  \PY{n}{test} \PY{o}{\PYZlt{}\PYZhy{}} \PY{n}{x\PYZus{}tst}
              \PY{p}{\PYZcb{}}
              
              \PY{n}{hr.train} \PY{o}{=} \PY{n}{x\PYZus{}tra}\PY{n}{[}\PY{p}{,}\PY{n+nf}{c}\PY{p}{(}\PY{l+s}{\PYZdq{}}\PY{l+s}{X2\PYZdq{}}\PY{p}{,}\PY{l+s}{\PYZdq{}}\PY{l+s}{X3\PYZdq{}}\PY{p}{,}\PY{l+s}{\PYZdq{}}\PY{l+s}{X4\PYZdq{}}\PY{p}{)}\PY{n}{]}
              \PY{n}{hr.labels.train} \PY{o}{=} \PY{n}{x\PYZus{}tra}\PY{n}{[}\PY{p}{,}\PY{l+s}{\PYZdq{}}\PY{l+s}{Y\PYZdq{}}\PY{n}{]}
              \PY{n}{hr.labels.train} \PY{o}{=} \PY{n+nf}{factor}\PY{p}{(}\PY{n}{hr.labels.train}\PY{p}{,} \PY{n}{levels}\PY{o}{=}\PY{n+nf}{c}\PY{p}{(}\PY{l+m}{1}\PY{p}{,}\PY{l+m}{2}\PY{p}{,}\PY{l+m}{3}\PY{p}{)}\PY{p}{)}
              \PY{n}{hr.train}\PY{o}{\PYZdl{}}\PY{n}{Y} \PY{o}{=} \PY{n}{hr.labels.train}
              \PY{n}{fitMulti}\PY{o}{=}\PY{n+nf}{tree}\PY{p}{(}\PY{n}{Y}\PY{o}{\PYZti{}}\PY{n}{X2}\PY{o}{+}\PY{n}{X3}\PY{o}{+}\PY{n}{X4}\PY{p}{,} \PY{n}{data}\PY{o}{=}\PY{n}{hr.train}\PY{p}{)}
              \PY{n}{labels} \PY{o}{=} \PY{n+nf}{factor}\PY{p}{(}\PY{n}{test}\PY{n}{[}\PY{p}{,}\PY{l+s}{\PYZdq{}}\PY{l+s}{Y\PYZdq{}}\PY{n}{]}\PY{p}{,}\PY{n}{levels}\PY{o}{=}\PY{n+nf}{c}\PY{p}{(}\PY{l+m}{1}\PY{p}{,}\PY{l+m}{2}\PY{p}{,}\PY{l+m}{3}\PY{p}{)}\PY{p}{)}
              \PY{n}{yprime}\PY{o}{=}\PY{n+nf}{predict}\PY{p}{(}\PY{n}{fitMulti}\PY{p}{,}\PY{n}{test}\PY{p}{,} \PY{n}{type}\PY{o}{=}\PY{l+s}{\PYZdq{}}\PY{l+s}{class\PYZdq{}}\PY{p}{)}
              \PY{n}{err} \PY{o}{=} \PY{n+nf}{mean}\PY{p}{(}\PY{n}{yprime}\PY{o}{!=}\PY{n}{labels}\PY{p}{)}
              
          \PY{p}{\PYZcb{}}
          \PY{n}{treeERRtrain.all}\PY{o}{\PYZlt{}\PYZhy{}}\PY{n+nf}{mean}\PY{p}{(}\PY{n+nf}{sapply}\PY{p}{(}\PY{l+m}{1}\PY{o}{:}\PY{l+m}{10}\PY{p}{,}\PY{n}{run\PYZus{}knn\PYZus{}fold}\PY{p}{,}\PY{n}{nombre}\PY{p}{,}\PY{l+s}{\PYZdq{}}\PY{l+s}{train\PYZdq{}}\PY{p}{)}\PY{p}{)}
          \PY{n}{treeERRtest.all}\PY{o}{\PYZlt{}\PYZhy{}}\PY{n+nf}{mean}\PY{p}{(}\PY{n+nf}{sapply}\PY{p}{(}\PY{l+m}{1}\PY{o}{:}\PY{l+m}{10}\PY{p}{,}\PY{n}{run\PYZus{}knn\PYZus{}fold}\PY{p}{,}\PY{n}{nombre}\PY{p}{,}\PY{l+s}{\PYZdq{}}\PY{l+s}{test\PYZdq{}}\PY{p}{)}\PY{p}{)}
          \PY{n+nf}{print}\PY{p}{(}\PY{n}{treeERRtrain.all}\PY{p}{)}
          \PY{n+nf}{print}\PY{p}{(}\PY{n}{treeERRtest.all}\PY{p}{)}
\end{Verbatim}


    \begin{Verbatim}[commandchars=\\\{\}]
[1] 0.1125
[1] 0.175

    \end{Verbatim}

    Como se puede ver, el modelo con árboles de decisión cuenta también con
cierto sobreaprendizaje, pero es el que mejores resultados obtiene de
todos. Aún así no lo utilizaremos para compararlo con el resto de
algoritmos ya que no se pide en la práctica.


    % Add a bibliography block to the postdoc
    
    
    
    \end{document}
